\chapter{Conclusão}
\label{ch.conclusao}

Este projeto mira em providenciar uma alternativa para consultar eventos e atividades culturais. 
Uma alternativa onde os usuários possam navegar e buscar por eventos culturais relevantes à eles, em uma única plataforma voltada para isso.
Uma alternativa one qualquer entidade organizadora de eventos possa divugar seus eventos, sem depender de mídias locais ou plataformas inadequadas.  

Por meio do trabalho desenvolvido, foi possível confirmar a factibilidade técnica desta plataforma, uma vez que foi implementado uma aplicação que cumpre os requisitos especifcados.
Lançar esta plataforma para uso geral, entretanto, é um desafio, que deve ser feito com o devido cuidado.

Tudo isso em prol de oferecer uma ferramenta que ajudasse as pessoas a encontrarem as atividades que tem interesse (ou até descobrir um novo!), e oferecer aos organizadores de eventos um público que aprecia os eventos que realizam. 
Desta forma, os organizadores de eventos teriam um aumento em seu público e em sua retenção, e as pessoas que frequentam esses eventos poderiam escolher o que mais lhe agrada.


% Este projeto mira ambiciosamente em se tornar a maneira mais usual de se consultar as atividades e eventos de uma determinada região, de maneira que os usuários utilizem-a como a primeira alternativa ao invés de navegar em páginas de redes sociais ou colunas de jornais locais. 
% A plataforma apresentaria dados de eventos de toda natureza, acompanhado de ferramentas para buscar, filtrar e organizá-los de maneira sucinta e simples. 
% Também agiria como uma plataforma para qualquer organização divulgar seu evento, não dependendo assim da redação dos jornais e mídias locais.
