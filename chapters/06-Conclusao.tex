\chapter{Conclusão}
\label{ch.conclusao}

Este projeto mira ambiciosamente em se tornar a maneira mais usual de se consultar as atividades e eventos de uma determinada região, de maneira que os usuários utilizem-a como a primeira alternativa ao invés de navegar em páginas de redes sociais ou colunas de jornais locais. A plataforma apresentaria dados de eventos de toda natureza, acompanhado de ferramentas para buscar, filtrar e organizá-los de maneira sucinta e simples. Também agiria como uma plataforma para qualquer organização divulgar seu evento, não dependendo assim da redação dos jornais e mídias locais.

Tudo isso em prol de oferecer uma ferramenta que ajudasse as pessoas a encontrarem as atividades que tem interesse (ou até descobrir um novo!), e oferecer aos organizadores de eventos um público que aprecia os eventos que realizam. Desta forma, os organizadores de eventos teriam um aumento em seu público e em sua retenção, e as pessoas que frequentam esses eventos poderiam escolher o que mais lhe agrada.

Embora seja desafiador quebrar o status quo, incentivar usuários e organizações a aderirem à plataforma e concorrer com as alternativas, existe um potencial desde projeto se tornar uma ferramenta verdadeiramente útil, onde o lucro não é necessariamente o foco, mas sim uma consequência de um bom design que leva em consideração seus usuários.
