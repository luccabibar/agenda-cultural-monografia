\chapter{Conclusão}
\label{ch.conclusao}

Este trabalho mira em providenciar uma alternativa para consultar eventos e atividades culturais. 
Uma alternativa onde os usuários possam navegar e buscar por eventos culturais relevantes à eles, em uma única plataforma voltada para isso.
Uma alternativa one qualquer entidade organizadora de eventos possa divugar seus eventos, sem depender de mídias locais ou plataformas inadequadas.  

Por meio do trabalho desenvolvido, foi possível confirmar a factibilidade técnica desta plataforma, uma vez que foi implementado uma aplicação que cumpre os requisitos especifcados.
Lançar esta plataforma para uso geral, entretanto, é um desafio, que deve ser feito com o devido cuidado.

Tudo isso em prol de oferecer uma ferramenta que ajudasse as pessoas a encontrarem as atividades que tem interesse — ou até descobrir um novo — e oferecer aos organizadores de eventos um público que aprecia os eventos que realizam. 
Desta forma, os organizadores de eventos teriam um aumento em seu público e em sua retenção, e as pessoas que frequentam esses eventos poderiam escolher o que mais lhe agrada.

% Este projeto mira ambiciosamente em se tornar a maneira mais usual de se consultar as atividades e eventos de uma determinada região, de maneira que os usuários utilizem-a como a primeira alternativa ao invés de navegar em páginas de redes sociais ou colunas de jornais locais. 
% A plataforma apresentaria dados de eventos de toda natureza, acompanhado de ferramentas para buscar, filtrar e organizá-los de maneira sucinta e simples. 
% Também agiria como uma plataforma para qualquer organização divulgar seu evento, não dependendo assim da redação dos jornais e mídias locais.

\section{Dificuldades encontradas}
\label{sec.dificuldades-encontradas}

Desenvolvimento de software é um processo que dificilmente ocorre sem intempéries. 

Um desafio relevante encontrado foi em relação à tecnologia utilizada para desenvolver o componetne \emph{backend} da aplicação.
Inicialmente, foi escalado a linguagem de programação Python, junto do \emph{framework} Flask, para cumprir este papel.
O motivo da escolha se deve à facilidade inicial de desenvolvimento com estas tecnologias.
Python é uma linguagem fracamente tipada, o que torna o desenvolvimento veloz, a princípio.
De maneira similar, Flask é um \emph{framework} inicialmente fácil de se entender, também habilitando o desenvolvimento ágil da aplicação.
Um protótipo inicial da aplicação utilizando estas tecnologias chegou a ser desenvolvido.

Entretando, essas qualidades deixaram de valer a pena conforme o projeto crescia.
A ausência de tipos tornava o desenvolvimento cada vez mais lento, uma vez que não se tinha segurança sobre a maneira como os dados transitavam entre os componentes, o que tornava o desenvolvimento mais difícil e implicava em diversas verificações de integridade de dados.
Adicionalmente, a abordagem da linguagem para importar outros scripts da aplicação tornaram a organização e estruturação da aplicação incômodas, requerindo burocracia adicional para transformar arquivos em pacotes importáveis.

Após o desenvolvimento do protótipo, foi ponderado com muita cautela a possibilidade de descartar o código já desenvolvido em python, para implementar um novo \emph{backend}, utilizando uma tecnologia diferente.
A principal solução cogitada foi a linguagem Java, junto ao \emph{framework} Springboot.
As restrições e burocracias trazidas por esta solução são justamente sua maior força: elas incentivam um desenvolvimento estruturado e consistente, garantindo que seja possível navegar e entender projetos complexos.
A linguagem Java, por sua vez, é uma linguagem fortemente tipada, o que garante a consistência de dados entre os componentes.
Adicionalmente, o springboot oferece soluções para acesso ao banco de dados e para autenticação por meio de filtro de requisições, o que tornou o desenvolvimento destes processos mais rápido e confiável. 

Depois deste período de estudo, seguido de um breve período de aprendizagem, foi consumada a escolha desta ferramenta como a que seria utilizada para desenvolver o \emph{backend} do projeto.
Apesar do contratempo de ter de reimplementar uma porção do trabalho, somado ao processo de aprendizagem da tecnologia, o desenvolvimento deste componente seguiu de maneira eficiente e consistente, permitindo a sua implementação satisfatória.

O único percalço durante esta etapa foi durante o desenvolvimento do mecanismo de autenticação por filtro de requisições.
Para sua implementação, foi necessário estudar sobre o funcionamento da ferramenta, de maneira mais profunda comparado à outras funcionalidades do \emph{framework}.
Entretanto, uma vez que se entende como trabalhar com esta solução, os filtros se tornam uma ferramenta poderosa a ser utilizada no desenvolvimento de qualquer aplicação estilo REST. 

\section{Trabalhos futuros}
\label{sec.trabalhos-futuros}

Embora o trabalho desenvolvido tenha atingido um estado satisfatório, ainda há espaço para desenvolver mais ideias, implementar melhorias e novas funcionalidades.

Nenhum sistema opera em um vácuo.
Isso significa que, para um eventual lançamento da plataforma desenvolvida, convém realizar uma análise de mercado profunda, para determinar a melhor estratégia de lançamento possível.
Isto inclui também uma campanha de marketing para a plataforma, para alcançar usuários e entidades organizadoras, bem como uma estratégia de monetização, levando em consideração os custos operacionais e maneiras de adquirir receita para sustentar sua operação e manutenção.

Outro componente importante para o funcionamento da plataforma são os termos de uso, documento que nortearia os Moderadores ao analisar os eventos da plataforma.
Ele deve ser elaborado com cuidado, a fim de garantir que o conteúdo da aplicação tenha qualidade, e que ela não seja pervertida por usuários maliciosos.

Conforme discutido neste trabalho, o desenvolvimento de uma versão que seja executável em aparelhos celulares e tablets tornaria a plataforma ainda mais acessível e prática.
Estar disponível neste tipo de aparelho tornaria a plataforma disponível a um público bem maior, e também possibilitaria acessá-la em momentos nos quais um computador com um e navegador e acesso à internet não estão facilmente acessíveis (como, por exemplo, enquanto se está na rua).

Visto que a aplicação é dividida entre cliente e servidor, seria necessário desenvolver apenas um novo componente cliente, que execute em dispositivos móveis, responsável por acessar os dados do servidor e apresentá-los ao usuário.
Adicionalmente, as tecnologias utilizadas no desenvolvimento do cliente para navegador web permitem que uma boa porção do código elaborado seja reaproveitado no desenvolvimento de um novo cliente, assim reduzindo o custo de desenvolvimento.

No que se refere à acessibilidade, a plataforma ainda não está de acordo com normas regulamentadoras sobre acessibilidade digital.
A fim de tornar a plataforma devidamente acessível à pessoas com deficiência, devem ser implementadas medidas que tornem o \emph{website} acessível. 
Estas medidas incluem, mas não estão limitadas a: um modo de alto contraste; suporte para libras; uso correto do texto alternativo; e legibilidade otimizada.

A principal melhoria ao sistema que poderia ser desenvolvida seria o emprego de uma integração com o Google Maps.
O serviço de mapas oferecido pela Google permite a interação com um mapa virtual, onde é possível navegar, escolher endereços, marcar e exibir lugares específicos.
A implementação desta integração consistiria em recolher um lugar no mapa ao cadastrar um novo evento, correspondente ao endereço do mesmo, e também na exibição deste lugar na página de eventos.
Desta maneira, os usuários podem conferir o endereço do evento em um mapa, definirem uma rota de sua localização ao local do evento, e todas as outras funcionalidades que o Google Maps já oferece.

Outras funcionalidades envolvendo o serviço de mapas inclúem a filtragem de eventos por geolocalização na página de busca, e também uma nova página, contendo um mapa onde se exibem eventos que acontecem numa determinada região.

% AFERIR HOMEPAGE
Embora a página de agenda cumpra este papel, o desenvolvimento de uma \emph{homepage}, com o intuito de ser a página principal da aplicação, pode torná-la mais agradavel de se utilizar. 
A mesma poderia exibir eventos em destaque, e eventos relevantes ao usuário, levando em conta a data, hora e região de acesso. 
Ter uma página assim como a primeira que um novo usuário encontraria deve transmitir a ideia de como a aplicação e seus eventos funcionam, tornando o \emph{onboarding} mais suave e aumentando as chances do usuário voltar a acessar a aplicação.

Ao realizar o cadastro, a conta do usuário é automaticamente ativada. 
Idealmente, seria implementado um sistema de verificação de email para realizar o login, bem como um sistema de recuperação de senha, caso o usuário a esqueça.

Ao cadastrar um evento, seria ideal recolher a classificação indicativa do mesmo, e utilizar este parâmetro como filtro nas páginas de busca de eventos e agenda de eventos. 
Esta é uma informação importante especialmente para pais e responsáveis, e ter mais este atributo relacionado aos eventos tornaria a experiência deste grupo ainda melhor.

Em relação aos atributos de um evento, permitir o upload de mais imagens, imagens maiores, e títulos e descrições mais longas podem tornar o uso das páginas de criação e edição de eventos mais flexível e agradável aos organizadores, tornando os eventos mais expressivos e detalhados.

Tanto a solução de armazenamento de imagens, quanto o mecanismo de designação de moderadores a eventos foram desenvolvidas com a intenção de serem melhoradas conforme a aplicação crescer e escalar.
Mais especificamente, o armazendamento de imagens é feito no próprio servidor, mas o serviço que gerencia o armazenamento pode ser substituído por um novo serviço, que utilize de armazenamento em nuvem, por exemplo.
De maneira similar, a função que atribui um moderador à um evento o faz de maneira aleatória, mas pode ser substituída para realizar uma escolha mais informada.
Estas funcionalidades foram desenvolvidas tendo em mente o princípio do baixo acoplamento, o que significa que podem ser alteradas sem comprometer o funcionamento de outras partes da aplicação.
Isso não significa que estas duas funcionalidades foram implementadas de maneira inadequada, mas sim que elas funcionam para uma escala específica da aplicação.

Ao longo da aplicação, cabem diversas pequenas otimizações, que tornariam a experiência de utilização mais agradável.
Estas inclúem: uma maneira de pré-visualizar a página de evento, a partir das páginas de novo evento e editar evento; deixar registrado preferências de usuários com filtros na página de agenda de eventos, a fim de mostrar eventos mais relevantes; e oferecer uma opção de ignorar eventos já passados na página de busca de eventos.
