\chapter{Descrição do Projeto}
\label{ch.descricao-projeto}

O projeto consiste em uma aplicação, estruturada na arquitetura Cliente-Servidor, onde a principal entidade é um “Evento”, e os três principais atores que com ela interagem são os “Usuários”, “Organizadores” e “Moderadores”. Toda a aplicação gira em torno desses Eventos, como eles são criados, organizados e consumidos, portanto entender como ele se relaciona com os atores é o suficiente para entender a estrutura e funcionamento da aplicação.

Um Evento é, basicamente, um conjunto de dados correspondente a um evento cultural (tal qual um show, peça ou festival), contando com dados como nome, categoria, descrição, organizadores, data, local e uma imagem promocional. Cada Evento tem uma página própria, onde todas as informações referentes são apresentadas com clareza. A \autoref{fig:tipo-evento} representa como o Evento é estruturado, definindo então o Tipo Evento. Relacionado ao Evento, há uma lista de atualizações denominada AtualizacaoEvento, que armazena pequenas atualizações de texto sobre o Evento.

\begin{figure}[H]
    \centering
    \caption[Tipo Evento]{Tipo Evento}
    \includegraphics[width = 0.6\linewidth]{figs/03-descricao-projeto/tipo-evento.png}\\
    \fonte{Elaborada pelo autor.}
    \label{fig:tipo-evento}
\end{figure}

% TODO:  usuario -> pessoa

O ator denominado Pessoa representa a interação mais básica com a aplicação: um usuário, que não precisa estar logado, que acessa a plataforma para procurar dados sobre eventos culturais em sua região. A página principal da aplicação conta com um calendário semanal, onde são apresentados os Eventos que estarão ocorrendo. Também é possível acessar uma página onde há um calendário diário dos Eventos. Também existe uma página de busca, que permite filtrar e buscar por Eventos específicos, baseado em texto, data, categoria e região. É possível, a qualquer momento, clicar em um Evento para navegar à sua página correspondente.

O ator denominado Organizador representa uma organização que gerencia eventos culturais, e é o responsável por criar os Eventos (isto é, preencher um formulário com os dados de um evento cultural). Ao contrário do Usuário, o Organizador deve realizar um login, de maneira que seja possível estabelecer um contato. Seu papel é de criar os Eventos, de maneira que entrem no sistema, a entidade seja criada, e possam ser consumidos pelos usuários. Este processo é realizado por meio de uma página que age como um formulário, coletando os dados referentes à entidade. Uma Organização pode ainda editar um Evento, adicionar uma atualização a ele, ou excluí-los caso necessário.

O último ator é o Moderador, e ele assume uma função administrativa na plataforma: uma vez que Eventos criados por Organizações não são imediatamente visíveis ao público, seu trabalho é de analisar, aprovar ou reprovar Eventos criados, de maneira que nenhum deles viole os termos de uso da plataforma (como, por exemplo, spam). Um Evento que é aprovado por um Moderador se torna visível aos Usuários, ao passo que um reprovado é restringido, mas ainda pode ser editado pelo Organizador referente. Dado a natureza do papel que este ator exerce, os moderadores são escolhidos pelos gestores da plataforma. 

O comportamento de cada um dos atores em relação à aplicação estão descrito na \autoref{fig:use-case-usuario} e \autoref{fig:use-case-moderador-organizador}, representados por meio de diagramas de caso de uso:

\begin{figure}[H]
    \centering
    \caption[Caso de Uso - Usuário]{Caso de Uso - Usuário}
    \includegraphics[width = 0.7\linewidth]{figs/03-descricao-projeto/use-case-usuario.png}\\
    \fonte{Elaborada pelo autor.}
    \label{fig:use-case-usuario}
\end{figure}

\begin{figure}[H]
    \centering
    \caption[Caso de Uso - Organizador e Moderador]{Caso de Uso - Organizador e Moderador}
    \includegraphics[width = 0.7\linewidth]{figs/03-descricao-projeto/use-case-moderador-organizador.png}\\
    \fonte{Elaborada pelo autor.}
    \label{fig:use-case-moderador-organizador}
\end{figure}

Uma vez que definimos tanto o tipo Evento, quanto os atores do sistema, podemos entender como eles se relacionam, por meio de um diagrama de classe, representado  na \autoref{fig:diagrama-classe}.

\begin{figure}[H]
    \centering
    \caption[Diagrama de Classe]{Diagrama de Classe}
    \includegraphics[width = 1.0\linewidth]{figs/03-descricao-projeto/diagrama-classe.png}\\
    \fonte{Elaborada pelo autor.}
    \label{fig:diagrama-classe}
\end{figure}

De maneira sucinta, podemos dividir a plataforma nas seguintes telas principais, referentes aos seguintes atores:

% \begin{center}
%     \begin{tabular}{|c|c|c|}
%         \hline
%         Usuário & Organização & Moderador \\ 
%         \hline
%         Página Principal & \makecell{Listagem de Eventos \\ (para Gestão)} & \makecell{Listagem de Eventos \\ (para Análise)} \\  
%         Agenda Semanal & Criação de Eventos & \makecell{Página de Evento \\ (para Análise)} \\    
%         Agenda Mensal & \makecell{Página de Evento \\ (para Gestão)} &  \\    
%         Agenda Diária &   &  \\    
%         Busca de Eventos &  &  \\    
%         Página de Evento &  &  \\
%         \hline
%     \end{tabular}
% \end{center}

\begin{quadro}[H]
    \caption{Telas da aplicação separadas por tipo de Usuário.}
    % \resizebox{\textwidth}{!}{%
        \begin{tabular}{|c|c|c|}
            \hline
            \textbf{Usuário} & \textbf{Organização} & \textbf{Moderador} \\ 
            \hline
            Página Principal & \makecell{Listagem de Eventos \\ (para Gestão)} & \makecell{Listagem de Eventos \\ (para Análise)} \\  
            Agenda Semanal & Criação de Eventos & \makecell{Página de Evento \\ (para Análise)} \\    
            Agenda Mensal & \makecell{Página de Evento \\ (para Gestão)} &  \\    
            Agenda Diária &  &  \\    
            Busca de Eventos &  &  \\    
            Página de Evento &  &  \\
            \hline
        \end{tabular}
    % }
\fonte{Elaborada pelo autor.}
\label{tabela:telas-aplicacao}
\end{quadro}

% \begin{quadro}[H]
% \caption{Telas da aplicação separadas por tipo de Usuário.}
% \resizebox{\textwidth}{!}{%
% \begin{tabular}{|
% >{\columncolor[HTML]{FFFFFF}}l |
% >{\columncolor[HTML]{FFFFFF}}l |
% >{\columncolor[HTML]{FFFFFF}}l |}
% \hline
% \multicolumn{1}{|c|}{\cellcolor[HTML]{FFFFFF}\textbf{Usuário}} &
% \multicolumn{1}{c|}{\cellcolor[HTML]{FFFFFF}\textbf{Organizador}} &
% \multicolumn{1}{c|}{\cellcolor[HTML]{FFFFFF}\textbf{Moderador}} \\ \hline
% Página Principal & \makecell{Listagem de Eventos \\ (para Gestão)} & \makecell{Listagem de Eventos \\ (para Análise)} \\  
% Agenda Semanal & Criação de Eventos & \makecell{Página de Evento \\ (para Análise)} \\    
% Agenda Mensal & \makecell{Página de Evento \\ (para Gestão)} &  \\    
% Agenda Diária &   &  \\    
% Busca de Eventos &  &  \\    
% Página de Evento &  &  \\ \hline
% \end{tabular}%
% }
% \fonte{Elaborada pelo autor.}
% \label{tabela:comparacao-datasets}
% \end{quadro}

\section{Tecnologias Utilizadas}
\label{sec.tecnologias-utilizadas}
    
Dado a natureza do projeto, seus objetivos e seus requisitos, é possível definir como ele será implementado, assim como as tecnologias escolhidas para tal. 

Em termos de arquitetura, a mais adequada é a Cliente-Servidor: ela permite que um \emph{Backend} (denominado Servidor) receba, armazene, processe, gerencie e sirva dados, de acordo com pedidos realizados por um \emph{Frontend} (denominado Cliente), através de requisições HTTP feitas através da internet. Esta arquitetura é vantajosa pois desacopla o \emph{Frontend} do \emph{Backend}, permitindo que sejam desenvolvidos em paralelo, com dependências reduzidas. O único co-requisito entre as duas partes é que elas sejam capazes de se comunicar por meio de requisições HTTP, o que não é uma funcionalidade incomum. Adicionalmente, ela permite que diversos Clientes diferentes se comuniquem com o mesmo Servidor, flexibilizando a natureza do Cliente, que pode ser desde uma aplicação  web, até um aplicativo mobile.

% reescrever sabosta fodase piton


É o tipo de aplicação que cumpre de forma concisa e confiável o que se espera para este componente: uma maneira padronizada dos Clientes acessarem o Servidor por meio de requisições HTTP, de modo que o Servidor consiga as responder de acordo; amplo apoio para aplicações REST, como controladores REST; e acesso ao banco de dados via JDBC.

Para o \emph{Backend} do projeto, foi empregado a linguagem de programação Java, junto ao \emph{framework} Springboot. 
Ele permitiu o desenvolvimento, de maneira robusta e confiável, de um meio para os Clientes acessarem as funcionalidades do Servidor, de maneira padronizada por uma API REST.
O \emph{framework} oferece amplo apoio para aplicações REST, como controladores REST, filtros de requisições HTTP, e acesso ao banco de dados via JDBC. Utilizando o filtro de requisições, foi possível implementar uma boa solução de autenticação de usuários via JWT, que foi a abordagem escolhida para esta tarefa. 

Para armazenar os dados de usuários de dos eventos, foi escolhido um banco de dados relacional tradicional, o PostgreSQL, devido a sua fácil utilização tanto com o Springboot quanto com a própria arquitetura Cliente-Servidor.

Para o \emph{Frontend}, o Cliente foi implementado como uma aplicação Web, a princípio. Este tipo de aplicação funciona bem na arquitetura Cliente-Servidor, visto que a própria natureza dela implica numa ênfase em requisições HTTP. Utilizando o \emph{framework} Angular, que emprega a linguagem TypeScript, é possível desenvolver uma aplicação robusta para navegadores web. Adicionalmente, aplicações utilizando Angular costumam se comportar bem em navegadores para aparelhos celulares, tornando-se assim uma escolha flexível.

Ambas as linguagens escolhidas são fortemente tipadas, e oferecem apoio à abordagem de programação orientada à objeto. Isto garante a plena comunicação entre cada componente da aplicação, e garante que ela se comporte exatamente como esperado, facilitando o desenvolvimento do projeto.

Deve-se notar que boa parte do público em potencial do serviço utilizaria aparelhos celulares para acessá-lo. Com isso em mente, também é possível desenvolver um aplicativo de celular que agisse como um Cliente. Soluções PWAs, ou \emph{Progressive Web Apps}, são as ideais, visto que permitem desenvolver aplicativos mobile utilizando ferramentas usadas no desenvolvimento web tradicional. No caso do Ionic, um \emph{framework} PWA, é empregado o próprio Angular, assim justificando ainda mais a escolha deste, visto que isto tornaria as aplicações web e mobile semelhantes, e reduziria o atrito no desenvolvimento paralelo das duas.
    
\section{Procedimentos de Validação}
\label{sec.procedimentos-validacao}

% TODO: duvida testes
    
Para garantir o funcionamento do sistema, ele será testado durante e após o seu desenvolvimento, empregando a estratégia da Pirâmide de Testes, onde cada nível da pirâmide corresponde a um tipo de teste a ser executado na aplicação. Com essa estrutura, é possível testar a aplicação de maneira integral, garantindo que tudo esteja funcionando em todos os níveis possíveis.

A base da pirâmide são os Testes Unitários, onde cada componente da aplicação é testado de maneira individual, verificando se o seu comportamento, em isolação, é o esperado. São os testes mais baratos de serem implementados e garantem o funcionamento individual de cada peça do sistema. Eles podem ser implementados a nível de componentes, classes ou funções.

O meio da pirâmide corresponde aos Testes de Integração, que irão averiguar como os componentes, já testados pelos Testes Unitários, se comunicam entre si. É nesta etapa que são testadas conexões e integrações entre serviços, clientes e servidores e diferentes blocos da aplicação, bem como o funcionamento de páginas como um todo. Portanto, são realizados a nível de páginas, ou de  \emph{endpoints} completos.

O topo desta pirâmide diz respeito aos Testes Ponta-a-Ponta (ou End to End, E2E), que são os mais complexos. Eles testam a aplicação seguindo uma abordagem de Caixa-Preta, ou seja, desconsiderando o funcionamento mecânico da aplicação e verificando apenas o sistema como ele é apresentado para o usuário. Em outras palavras, é um teste que envolve utilizar a aplicação tal qual ela foi projetada para, em busca de erros, bugs e inconsistências. Devido à sua natureza, testes E2E são realizados em nível de aplicação.
   
\section{Produto Desenvolvido}
\label{sec.produto-desenvolvido}

Durante o período em que ocorreram as disciplinas TCC 1 e 2, foi desenvolvido um Produto Mínimo Viável, e totalmente funcional, desta aplicação. Uma plataforma que permite às entidades organizadoras de eventos os divulgarem, e, aos usuários, encontrarem eventos relevantes para participarem.  

O componente \emph{backend} desenvolvido conta com diversos \emph{endpoints}, por onde é possível acessar as funcionalidades do Servidor. 
Os dois principais grupos de \emph{endpoints} são relacionados aos Eventos, e aos Usuários.
Por meio deles, é possível tomar todas as ações da aplicação: 
Criar uma conta e fazer login se referem aos \emph{endpoints} de Usuarios. Já buscar e consumir eventos, bem como criar, editar, moderar e excluí-los, são funcionalidades dos \emph{endpoints} de eventos.
O compoente backend também conta com um sistema de autenticação via JWT, garantindo que apenas os usuários adequados tem acesso à determinados recursos protegidos.

O componente \emph{frontend} desenvolvido age como o Cliente que acessa os dados do Servidor, implementando páginas na web que utilizam suas funcionalidades. 
Ele acessa essas funcionalidades por meio de requisições HTTP aos \emph{endpoints} do Servidor, por meio de um serviço que oferece funcionalidades de HTTP.
Como usuário, é possível procurar eventos por meio de filtros de busca, e ver seus detalhes além de conseguir visualizar todos os eventos que ocorrem em um determinado dia ou semana, por meio de uma agenda. 
Como um Organizador, é possível criar, editar e excluir eventos, e como um Moderador, é possível analisar e moderar os eventos.
É possível se identificar por meio de um sistema de login, que mantém o usuário ativo por meio de \emph{cookies} de navegador.

A identidade visual da adotada, embora não seja a final, reflete a visão da plataforma, visando ser sóbria, mas descontraída. Assim, ela emprega  um design claro e colorido. 

% TODO: figuras

As seguintes figuras, \autoref{fig:captura-busca} e \autoref{fig:captura-evento}, são capturas das respectivas telas, mencionadas no parágrafo acima.

\begin{figure}[H]
    \centering
    \caption[Captura - Tela de Busca]{Captura - Tela de Busca}
    \includegraphics[width = 1.0\linewidth]{figs/03-descricao-projeto/captura-busca.png}\\
    \fonte{Elaborada pelo autor.}
    \label{fig:captura-busca}
\end{figure}

\begin{figure}[H]
    \centering
    \caption[Captura - Tela de Evento]{Captura - Tela de Evento}
    \includegraphics[width = 1.0\linewidth]{figs/03-descricao-projeto/captura-evento.png}\\
    \fonte{Elaborada pelo autor.}
    \label{fig:captura-evento}
\end{figure}