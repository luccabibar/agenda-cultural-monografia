\chapter{Resultados}
\label{ch.resultados}

Durante o ciclo de desenvolvimento deste trabalho, foi desenvolvido a aplicação descrita nas seções anteriores, e que esta abordará.

Conforme especificado, a aplicação é composta de duas partes, o \emph{Backend}, que age como o Servidor, e o \emph{Frontend}, que age como o Cliente. 
Em um cenário usual, um servidor executa a porção \emph{Backend} da aplicação e serve o \emph{Frontend} ao usuário, que o executa em seu dispositivo.
Utilizando um navegador web, é possível acessar a aplicação desta maneira.

% TODO IBAGENS
% TODO TAMBEM: AFERIR EXISTENCIA DA HOMEPAGE
A página principal, exibida na \autoref{fig:agenda-diaria-terca}, é a primeira que um novo usuário vê, por padrão. 
É possível identificar um cabeçalho na página, também presente nas demais páginas, que contém alguns \emph{links} para navegação e uma barra de pesquisa.
Ao buscar por meio da barra de pesquisa, a aplicação navega à página de busca de evento, e realiaza uma busca usando o texto inserido.
Note também que, ao realizar login, os \emph{links} para as páginas de login e cadastro são substituídos por elementos relacionados ao usuário, como um botão para realizar logout, um \emph{link} para a página de perfil, e um \emph{link} para a página de criar evento, no caso de um usuário Organizador.

\begin{figure}[H]
    \centering
    \caption[Página principal]{Página principal}
    \includegraphics[width = 1.0\linewidth]{figs/capturas-aplicacao/agenda-diaria-terca.png}\\
    \fonte{Elaborada pelo autor.}
    \label{fig:agenda-diaria-terca}
\end{figure}

% TEXTO HOMEPAGE

A página agenda é composta de dois modos diferentes: agenda diária (modo inicial da página) e agenda semanal. 
É possível alternar entre eles por meio de botões.

No modo de agenda diária, exemplificado na \autoref{fig:agenda-diaria-sabado}, é apresentado a data corrente e uma grade, comportando todos os eventos que acontecem no dado dia, organizados por sua hora de início e dispostos ao longo de 24 colunas, correspondentes as 24 horas do dia.
Se alguma coluna contém uma grande quantidade de eventos, é exibido um botão, que leva para a página de busca de eventos, onde é realizado uma busca por todos os eventos realizados naquele horário.

\begin{figure}[H]
    \centering
    \caption[Agenda diária do dia 22/11/2025]{Agenda diária do dia 22/11/2025}
    \includegraphics[width = 1.0\linewidth]{figs/capturas-aplicacao/agenda-diaria-sabado.png}\\
    \fonte{Elaborada pelo autor.}
    \label{fig:agenda-diaria-sabado}
\end{figure}

Ao alternar para o modo de agenda semanal, como na \autoref{fig:agenda-semanal}, apresenta-se uma outra grade, com todos os eventos que ocorrem na semana da data corrente, inclindo o domingo da próxima semana, a fim de exibir todo o final de semana.
Nesta grade, cada coluna representa um dia da semana, e cada linha representa um horário.
Então, cada evento é exibido em uma célula correspondente a sua data e horário.
Similarmente à agenda diária, uma célula que contém eventos demais exibe um botão que leva para a página de busca de eventos, onde é realizado uma busca que mostra todos os eventos realizados naquela data e horário.
Adicionalmente, clicar na data de uma das colunas leva à agenda diária daquele dia. 

\begin{figure}[H]
    \centering
    \caption[Agenda semanal para a semana do dia 18/11/2025]{Agenda semanal para a semana do dia 18/11/2025}
    \includegraphics[width = 1.0\linewidth]{figs/capturas-aplicacao/agenda-semanal.png}\\
    \fonte{Elaborada pelo autor.}
    \label{fig:agenda-semanal}
\end{figure}

Clicar nos eventos apresentados na página de agenda dirige o usuário à página daquele evento. 
Há também duas caixas de seleção, que permitem filtrar os eventos exibidos nesta página por suas categorias ou regiões, conforme demonstrado na \autoref{fig:agenda-semanal-filtros}. 
Este filtro tem efeito em ambos os modos.

\begin{figure}[H]
    \centering
    \caption[Agenda semanal com os filtros aplicados]{Agenda semanal com os filtros aplicados}
    \includegraphics[width = 1.0\linewidth]{figs/capturas-aplicacao/agenda-semanal-filtros.png}\\
    \fonte{Elaborada pelo autor.}
    \label{fig:agenda-semanal-filtros}
\end{figure}

A página de buscas, exibida na \autoref{fig:busca}, permite buscar diretamente por eventos, passados e futuros, utilizando diversos filtros.
Além de navegar diretamente para esta página, é possível acessá-la tanto por meio da barra de pesquisa, no cabecalho da aplicação, quanto pela página de agenda.
Navegar para esta página por estes meios alternativos faz com que ela realize uma operação de busca com os parâmetros configurados pela pelo mecanismo que navegou para esta página.

\begin{figure}
    \centering
    \caption[Página de busca]{Página de busca}
    \includegraphics[width = 1.0\linewidth]{figs/capturas-aplicacao/busca.png}\\
    \fonte{Elaborada pelo autor.}
    \label{fig:busca}
\end{figure}

Na coluna esquerda desta página, localizam-se diversos filtros para busca: por texto (referente ao título e à descrição de um evento), por categoria, por região, por data, e por horário.
É possível observá-los sendo utilizados na \autoref{fig:busca-filtros}.
Cabe mencionar que os filtros de horário concernem apenas ao horário do evento, e não a sua data. 
Da mesma forma, os filtros de data não levam em consideração o horário de um evento.
Este comportamento é deliberado, para habilitar a filtragem de acordo com um intervalo de horários e um intervalo de dias simultaneamente, ao invés de um grande intervalo de um momento até outro.

\begin{figure}[H]
    \centering
    \caption[Página de busca, com filtros preenchidos]{Página de busca, com filtros preenchidos}
    \includegraphics[width = 1.0\linewidth]{figs/capturas-aplicacao/busca-filtros.png}\\
    \fonte{Elaborada pelo autor.}
    \label{fig:busca-filtros}
\end{figure}

Realizar uma busca nesta página exibe uma lista de eventos cujas características se enquadram nos parâmetros informados pelo usuário, conforme a figura \autoref{fig:busca-realizada}.
Clicar em qualquer evento navega para a página de evento, onde são exibidas informações referentes a ele.

\begin{figure}[H]
    \centering
    \caption[Página de busca, com busca realizada]{Página de busca, com busca realizada}
    \includegraphics[width = 1.0\linewidth]{figs/capturas-aplicacao/busca-realizada.png}\\
    \fonte{Elaborada pelo autor.}
    \label{fig:busca-realizada}
\end{figure}

A página de evento, acessível por meio de diversas outras páginas da aplicação, exibe as informações de um determinado evento, tal qual exemplificado na \autoref{fig:evento}.
Junto ao nome do evento, são exibidos sua descrição, categoria, uma imagem ilustrativa e um \emph{link} para informações adicionais, ao lado da data e horário de início e de término do evento, sua região, e seu endereço.
Também são exibidas as atualizações do dado evento, em forma de pequenos textos anexados ao evento descrevendo eventuais atualizações sobre o evento (como, por exemplo, um comunicado de troca de local).
É possível compartilhar o URL da página, direto do navegador, uma vez que a identificação do evento está presente nele. 

\begin{figure}[H]
    \centering
    \caption[Página de evento]{Página de evento}
    \includegraphics[width = 1.0\linewidth]{figs/capturas-aplicacao/evento.png}\\
    \fonte{Elaborada pelo autor.}
    \label{fig:evento}
\end{figure}

A última página acessível sem a necessidade realizar um login é justamente a página de login e cadastro.
Ambos os mecanismos de login e cadastro estão presentes nesta página, simultaneamente, a fim de evitar situações onde o usuário tenta acessar o cadastro e acaba acessando o login, ou vice-versa.
Ainda assim, a página evidencia o componente de login ou cadastro, a depender de qual \emph{link} foi clicado.

O componente de cadastro, mostrado na \autoref{fig:cadastro-pessoa}, contém um formulário, com os campos nome, email, e senha, e enviá-lo efetua a criação de um novo usuário no sistema, do tipo Pessoa. 
Também existe um botão, que altera o modo de cadastro para Organizador. 
Clicá-lo faz com este formulário apresente um novo campo para um CPF ou CNPJ (evidenciado na \autoref{fig:cadastro-organizador}), e enviá-lo cria um novo usuário do tipo Organizador, ao invés de Pessoa.

\begin{figure}[H]
    \centering
    \caption[Página de Login e Cadastro, com Cadastro em evidência, no modo de Pessoa]{Página de Login e Cadastro, com Cadastro em evidência, no modo de Pessoa}
    \includegraphics[width = 1.0\linewidth]{figs/capturas-aplicacao/cadastro-pessoa.png}\\
    \fonte{Elaborada pelo autor.}
    \label{fig:cadastro-pessoa}
\end{figure}

\begin{figure}[H]
    \centering
    \caption[Página de Login e Cadastro, com Cadastro em evidência, no modo de Organizador]{Página de Login e Cadastro, com Cadastro em evidência, no modo de Organizador}
    \includegraphics[width = 1.0\linewidth]{figs/capturas-aplicacao/cadastro-organizador.png}\\
    \fonte{Elaborada pelo autor.}
    \label{fig:cadastro-organizador}
\end{figure}

O componente de login, exibido na \autoref{fig:login}, e presente na mesma página, permite realizar a operação de autenticação.
Enviar este formulário com um e-mail e senha válidos tem como resposta os dados do usuário referente, bem como um \emph{Token}, para que se autenticar ao acessar funcionalidades restritas.
Uma vez realizado o login, é possível acessar novas páginas, que requerem que o usuário esteja autenticado.
O conjunto de páginas disponibilizado para cada usuário varia de acordo com o seu tipo.

\begin{figure}[H]
    \centering
    \caption[Página de Login e Cadastro, com Login em evidência]{Página de Login e Cadastro, com Login em evidência}
    \includegraphics[width = 1.0\linewidth]{figs/capturas-aplicacao/login.png}\\
    \fonte{Elaborada pelo autor.}
    \label{fig:login}
\end{figure}

A página de perfil está disponível para todos os usuários autenticados, porém se comporta de maneira ligeiramente diferente, a depender do tipo de usuário que a acessa.
Todos os usuários podem acessar a página e conferir dados do seu perfil.
Na \autoref{fig:perfil-pessoa}, se observa o perfil na visão de um usuário do tipo Pessoa.

\begin{figure}[H]
    \centering
    \caption[Página de perfil, para uma Pessoa]{Página de perfil, para uma Pessoa}
    \includegraphics[width = 1.0\linewidth]{figs/capturas-aplicacao/perfil-pessoa.png}\\
    \fonte{Elaborada pelo autor.}
    \label{fig:perfil-pessoa}
\end{figure}

Já na visão de um usuário do tipo Organizador, a página de perfil exibe uma lista, de todos os eventos criados por este Organizador, além dos dados do usuário.
A \autoref{fig:perfil-organizador} mostra como esta página fica para um usuário deste tipo.
Para cada evento, são exibidas suas informações básicas, o seu status, e botões, que permitem editar e exclui-lo.
É possível ainda filtrar essa lista por status, para facilitar a visualização.

\begin{figure}
    \centering
    \caption[Página de perfil, para um Organizador]{Página de perfil, para um Organizador}
    \includegraphics[width = 1.0\linewidth]{figs/capturas-aplicacao/perfil-organizador.png}\\
    \fonte{Elaborada pelo autor.}
    \label{fig:perfil-organizador}
\end{figure}

A página de edição de eventos, representada na \autoref{fig:atualizacao-edicao}, que permite editar ou atualizar um determinado evento, é acessível somente ao Organizador responsável pelo evento.
Esta página é dividida em dois componentes: um para adicionar atualizações ao evento, e outro para realizar edições no evento.
Eles estão agrupados nesta página porque ambos se referem a operações semelhantes: alterar o estado do evento.

\begin{figure}[H]
    \centering
    \caption[Página de atualização e edição]{Página de atualização e edição}
    \includegraphics[width = 1.0\linewidth]{figs/capturas-aplicacao/atualizacao-edicao.png}\\
    \fonte{Elaborada pelo autor.}
    \label{fig:atualizacao-edicao}
\end{figure}

O componente de atualização de evento possui um formulário que, preenchido com um título e uma descrição e envido, adiciona uma atualização ao evento referente, conforme exemplificado na \autoref{fig:atualizacao-preenchido}.
É possível visualizar tal atualização na página do evento.

\begin{figure}
    \centering
    \caption[Página de atualização e edição, com o formulário de atualização preenchido]{Página de atualização e edição, com o formulário de atualização preenchido}
    \includegraphics[width = 1.0\linewidth]{figs/capturas-aplicacao/atualizacao-preenchido.png}\\
    \fonte{Elaborada pelo autor.}
    \label{fig:atualizacao-preenchido}
\end{figure}

O componente de edição, também presente nesta página, possui um formulário, com campos de descrição, link de informações adicionais, imagem, data e hora de início e fim, região e endereço, todos referentes à atributos do evento. 
Durante o carregamento da página, estes campos são preenchidos com os dados originais deste evento.

A \autoref{fig:edicao-preenchido} ilustra um exemplo de uso deste componente:
Alterar algum campo deste formulário o marca com um tom de destaque, simbolizando que já nao reflete mais o valor original.
Os dados enviados por este formulário são constituídos apenas pelos campos que foram alterados pelo usuário, e vale notar que editar um evento define seu status como "em análise", sendo necessária uma nova análise de um Moderador para alterar seu status. 

\begin{figure}[H]
    \centering
    \caption[Página de atualização e edição, com o formulário de edição preenchido]{Página de edição e edição, com o formulário de atualização preenchido}
    \includegraphics[width = 1.0\linewidth]{figs/capturas-aplicacao/edicao-preenchido.png}\\
    \fonte{Elaborada pelo autor.}
    \label{fig:edicao-preenchido}
\end{figure}

A página de exclusão de eventos, representada na \autoref{fig:exclusao-evento}, também só é acessível ao Organizador deste evento, e contém apenas alguns botões: voltar, editar e excluir, que realizam ações sob um determinado evento. 
O botão de voltar direciona o usuário de volta à página de perfil, e o botão de editar o direcona à página de edição deste evento.
O botão de excluir, por sua vez, elimina este evento da aplicação, tornando-o inacessível.

\begin{figure}[H]
    \centering
    \caption[Página de exclusão de evento]{Página de exclusão de evento}
    \includegraphics[width = 1.0\linewidth]{figs/capturas-aplicacao/exclusao-evento.png}\\
    \fonte{Elaborada pelo autor.}
    \label{fig:exclusao-evento}
\end{figure}

Por fim, a última nova página a disponível a um Organizador é a de novo evento, exibida na \autoref{fig:novo-evento}. 
Ela é constituída de um formulário que coleta todos os dados necessários para a criação de um novo evento: seu nome nome, descrição, uma imagem ilustrativa, categoria, um link para informações adicionais, data e horário de início e término, região e endereço onde será realizado.
Ao preencher todos os campos de maneira válida e enviar o formulário, um novo evento é criado, mas não é imediatamente acessível: a princípio, possui um status de "em análise", e um Moderador deve analisá-lo e aprová-lo para que possa aparecer na aplicação normalmente.
Todo evento, ao ser criado, tem a ele um Moderador designado automaticamente.

\begin{figure}[H]
    \centering
    \caption[Página de novo evento]{Página de novo evento}
    \includegraphics[width = 1.0\linewidth]{figs/capturas-aplicacao/novo-evento.png}\\
    \fonte{Elaborada pelo autor.}
    \label{fig:novo-evento}
\end{figure}

Ao fazer login como Moderador, é possível acessar a página de perfil e conferir que, além dos dados de usuário, é também exibida uma lista de eventos, conforme demonstrado na \autoref{fig:perfil-moderador}.
Os eventos desta lista são aqueles que tem este usuário como Moderador designado.
Nesta lista, além de informações básicas de cada evento, está seu status, e um botão para o ato de análise do evento, caso o mesmo tenha o status "em análise".

\begin{figure}[H]
    \centering
    \caption[Página de perfil, para um Moderador]{Página de perfil, para um Moderador}
    \includegraphics[width = 1.0\linewidth]{figs/capturas-aplicacao/perfil-moderador.png}\\
    \fonte{Elaborada pelo autor.}
    \label{fig:perfil-moderador}
\end{figure}

A página de análise de eventos é acessível ao Moderador designado a um determinado evento, por meio de seu perfil.
Nela, visualiza-se todos os dados de um evento em análise, semelhante à página de eventos, de maneira que o Moderador possa analisá-lo.
Ao fim da página, há dois botões, que alteram o status deste evento para "aprovado" ou "reprovado", além de um terceiro botão que retorna ao perfil, tal qual mostrado na \autoref{fig:analise-evento}.
Aprovar um evento significa torná-lo visível para todos os usuários da aplicação, enquanto reprovar um evento faz com que ele não fique disponível a outros usuários, exceto ao Organizador — que pode editá-lo para que o evento entre em análise novamente.
O julgamento do status de um evento, realizado pelo Moderador, deve ser realizado de acordo com os termos de uso da plataforma.

\begin{figure}[H]
    \centering
    \caption[Página de análise de evento]{Página de análise de evento}
    \includegraphics[width = 1.0\linewidth]{figs/capturas-aplicacao/analise-evento.png}\\
    \fonte{Elaborada pelo autor.}
    \label{fig:analise-evento}
\end{figure}

Finalmente, existe uma página para qualquer URL não previsto pela aplicação, como se nota na \autoref{fig:notfound}.
Esta página também é exibida após uma tentativa de acessar um evento que não pode ser encontrado, ou quando uma página restrita, exemplificado na \autoref{fig:notauth}.

\begin{figure}[H]
    \centering
    \caption[Página para recursos não encontrados]{Página para recursos não encontrados}
    \includegraphics[width = 1.0\linewidth]{figs/capturas-aplicacao/notfound.png}\\
    \fonte{Elaborada pelo autor.}
    \label{fig:notfound}
\end{figure}

\begin{figure}[H]
    \centering
    \caption[Página para recursos de acesso restrito]{Página para recursos de acesso restrito}
    \includegraphics[width = 1.0\linewidth]{figs/capturas-aplicacao/notauth.png}\\
    \fonte{Elaborada pelo autor.}
    \label{fig:notauth}
\end{figure}