\chapter{Especificação}
\label{ch.especificacao-projeto}

O software desenvolvido consiste em uma aplicação, estruturada na arquitetura Cliente-Servidor, onde a principal entidade é um “Evento”, e os três principais atores que com ela interagem são as Pessoas, “Organizadores” e “Moderadores”.
Toda a aplicação gira em torno desses Eventos, como eles são criados, organizados e consumidos, portanto entender como ele se relaciona com os atores é o suficiente para entender a estrutura e funcionamento da aplicação.

Um Evento é, basicamente, um conjunto de dados correspondente a um evento cultural (tais quais shows, festas, eventos esportivos, etc.), contando com dados como nome, categoria, descrição, organizadores, data, local e uma imagem promocional.
Cada Evento tem uma página própria, em que todas as informações referentes são apresentadas com clareza. 
A \autoref{fig:classe-evento} representa como o Evento é estruturado, definindo então o Tipo Evento. 
Note que o Tipo Evento inclui uma lista de atualizações, denominada AtualizacaoEvento, que armazena pequenas atualizações de texto sobre o Evento.

\begin{figure}[H]
    \centering
    \caption[Tipo Evento]{Tipo Evento}
    \includegraphics[width = 0.6\linewidth]{figs/diagramas/diagrama-classe-evento.png}\\
    \fonte{Elaborada pelo autor.}
    \label{fig:classe-evento}
\end{figure}

% TODO 3.2 AFERIR EXISTENCIA DA HOMEPAGE
O ator denominado Pessoa representa a interação mais básica com a aplicação: um usuário,não necessariamente logado, que acessa a plataforma para procurar dados sobre eventos culturais. 
Desta forma, é possível navegar pelas páginas voltadas à exibição de eventos, visto que elas não requerem que o usuário faça login. % A página principal conta com um banner exibindo eventos em destaque, e uma seleção de eventos relevantes separados por categoria.
A página agenda oferece uma exibição dos eventos ao longo de um dia ou de uma semana.
Na agenda diária, os eventos são organizados pelo seu horário de início.
Já na agenda semanal, os eventos são organizados pela sua data marcada, mas também pelo seu horário de início, dispondo-os em uma grade, assim tomando a forma de uma agenda. 
Também existe uma página de busca, que permite filtrar e buscar por Eventos específicos, baseado em texto, data, horário, categoria e região. 
É possível, a qualquer momento, clicar em um Evento para navegar à sua página correspondente, onde são exibidas todas os seus detalhes.

O ator denominado Organizador representa uma entidade organizadora de eventos culturais, e é o responsável por criar e gerir os Eventos. 
Ao contrário do Usuário, deve realizar seu cadastro na página de cadastro, de maneira que seja possível estabelecer um contato. 
Uma vez feito login (utilizando dados do cadastro), é possível acessar páginas referentes às funções de Organizador, além das já disponíveis para Pessoas.

A página de novo evento permite preencher um formulário com dados de uma atividade cultural, resultando na criação de um novo evento, que será consumido pelos demais usuários.
Também é possível acessar uma página de perfil, onde são exibidos dados deste ator, e também uma lista dos eventos que gerencia.
Nesta lista, é possível ver todos os eventos deste denominado ator, verificar o status (se foi Aprovado, Reprovado ou se está Em Análise) de cada um, e ainda navegar para a página de edição e exclusão respectiva a cada evento. 
A página de edição de evento permite ao Organizador tomar duas ações relativas ao Evento: Editar os seus dados, ou adicionar uma Atualização de texto ao evento. 
Por fim, a página de exclusão de evento oferece a remoção de um determinado evento do sistema, caso seu Organizador assim o queira.

O terceiro ator é o Moderador, e ele assume um papel administrativo na plataforma: realiza a análise antes de um Evento recém criado chegar ao público, aprovando-o ou não, conforme com os termos de uso da plataforma (como, por exemplo, evitar spam)
Dado a natureza do papel deste ator, os Moderadores não realizam cadastro como os demais, mas são escolhidos pelos gestores da plataforma.
Na página de perfil do Moderador, há uma lista de eventos, bem como um botão que o leva à página de análise de evento.
Os eventos listados nesta página são aqueles designados à este Moderador, que estão em estado de análise, e devem ser analisados e ter seu status atualizado.
A página de análise de evento, que apenas o Moderador de um evento tem acesso, apresenta os dados de um determinado Evento em estado de análise, e permite que se atualize o seu status para 'Aprovado' ou 'Reprovado', a depender do julgamento.
Um Evento que é aprovado por um Moderador se torna visível aos usuários, ao passo que um reprovado é restringido, mas ainda pode ser editado para uma nova tentativa pelo Organizador responsável. 


O comportamento de cada um dos atores em relação à aplicação estão descrito na \autoref{fig:use-case-usuario} e \autoref{fig:use-case-moderador-organizador}, representados por meio de diagramas de caso de uso:

% \begin{figure}[H]
%     \centering
%     \caption[Caso de Uso - Usuário]{Caso de Uso - Usuário}
%     \includegraphics[width = 0.7\linewidth]{figs/diagramas/diagrama-casouso-usuario.png}\\
%     \fonte{Elaborada pelo autor.}
%     \label{fig:use-case-usuario}
% \end{figure}

% TODO 3.2 CONT
\begin{figure}[H]
    \centering
    \caption[Diagrama de Caso de Uso - Usuário]{Diagrama de Caso de Uso - Usuário}
    \includegraphics[width = 0.7\linewidth]{figs/diagramas/diagrama-casouso-usuario-homeless.png}\\
    \fonte{Elaborada pelo autor.}
    \label{fig:use-case-usuario}
\end{figure}

% TODO diagrama falta exlcuir
\begin{figure}[H]
    \centering
    \caption[Diagrama de Caso de Uso - Organizador e Moderador]{Diagrama de Caso de Uso - Organizador e Moderador}
    \includegraphics[width = 0.7\linewidth]{figs/diagramas/diagrama-casouso-orgmod.png}\\
    \fonte{Elaborada pelo autor.}
    \label{fig:use-case-moderador-organizador}
\end{figure}

Uma vez que definimos tanto o tipo Evento, quanto os atores do sistema, podemos entender como eles se estruturam e se relacionam, e sintetizar em um diagrama Entidade-Relacionamento, representado na \autoref{fig:diagrama-banco}.

\begin{figure}[H]
    \centering
    \caption[Diagrama Entidade-Relacionamento]{Diagrama Entidade-Relacionamento}
    \includegraphics[width = 1.0\linewidth]{figs/diagramas/diagrama-banco.png}\\
    \fonte{Elaborada pelo autor.}
    \label{fig:diagrama-banco}
\end{figure}

Também é possível definir um diagrama de classes, que representa como estes dados vão ser implementados na aplicação, representado na \autoref{fig:diagrama-classe}. 
Vale notar que algumas das entidades neste diagrama se referem à outros aspectos da aplicação além dos descritos nesta seção. 
Por exemplo, as classes \emph{JWTUser} e \emph{UsuarioAutenticado}, bem como seus relacionamentos com \emph{UsuarioAutenticado}, tomam esta forma em razão da estratégia de autenticação de usuário adotada pela plataforma. 

\begin{figure}[H]
    \centering
    \caption[Diagrama de Classe]{Diagrama de Classe}
    \includegraphics[width = 1.0\linewidth]{figs/diagramas/diagrama-classe.png}\\
    \fonte{Elaborada pelo autor.}
    \label{fig:diagrama-classe}
\end{figure}

De maneira sucinta, é possível dividir a plataforma nas seguintes telas principais, referentes aos seguintes atores:

% \begin{center}
%     \begin{tabular}{|c|c|c|}
%         \hline
%         Usuário & Organização & Moderador \\ 
%         \hline
%         Página Principal & \makecell{Listagem de Eventos \\ (para Gestão)} & \makecell{Listagem de Eventos \\ (para Análise)} \\  
%         Agenda Semanal & Criação de Eventos & \makecell{Página de Evento \\ (para Análise)} \\    
%         Agenda Mensal & \makecell{Página de Evento \\ (para Gestão)} &  \\    
%         Agenda Diária &   &  \\    
%         Busca de Eventos &  &  \\    
%         Página de Evento &  &  \\
%         \hline
%     \end{tabular}
% \end{center}

\begin{quadro}[H]
    \caption{Telas da aplicação separadas por tipo de Usuário.}
    % \resizebox{\textwidth}{!}{%
        \begin{tabular}{|c|c|c|}
            \hline
            \textbf{Pessoa} & \textbf{Organizador} & \textbf{Moderador} \\ 
            \hline
            \makecell{Agenda Semanal e\\Agenda diária}  & \makecell{Perfil\\Organizador}            & \makecell{Perfil\\Moderador}                 \\  
            Página de Busca                             & \makecell{Criar\\Evento}                  & \makecell{Analisar\\Evento}                  \\    
            Página de Evento                            & \makecell{Atualizar e\\Editar Evento}     &                                              \\    
                                                        & \makecell{Excluir\\Evento}                &                                              \\    
            \hline
        \end{tabular}
    % }
\fonte{Elaborada pelo autor.}
\label{tabela:telas-aplicacao}
\end{quadro}