\chapter{Soluções Existentes}
\label{ch.solucoes-existentes}

Já existem alguns outros meios que tentam agregar as atividades culturais da região de Bauru. 
Entretanto, nenhuma delas apresenta o conjunto de características com a completude que o presente trabalho visa implementar. 
Seguem exemplos:

A Agenda Social Bauru \cite{agenda-socialbauru} é uma página da web que reúne diversos programas culturais da cidade, dirigida pela Social Bauru, uma mídia online focada em divulgar atividades e atrações. 
Conforme representada na \autoref{fig:agenda-socialbauru}, a página contém diversos eventos, organizados por categorias e separados por dia, ao longo de uma lista em texto corrido. 
Entretanto, a coluna é restrita (visto que é escrita por um redator da agência, não cumprindo então a função de uma plataforma aberta), e é orientada, sobretudo, por uma intenção publicitária. 
Além disso, é possível visualizar o conteúdo apenas em forma de lista.

\begin{figure}
    \centering
    \caption[Página da agenda Social Bauru]{Página da agenda Social Bauru}
    \includegraphics*[width = 1.0\linewidth, height = 0.9\textheight]{figs/capturas-solucoes-existentes/agenda-socialbauru.png}\\
    \fonte{\cite{agenda-socialbauru}}
    \label{fig:agenda-socialbauru}
\end{figure}

Outra alternativa é a Agenda JC \cite{agenda-jcnet}, da JCNet. 
Ela é mais completa, mas ainda é restrita, por ser escrita por uma redação. 
A página, exibida na \autoref{fig:agenda-jcnet}, lista o horário de funcionamento de diversos estabelecimentos da cidade, como restaurantes, bares e lanchonetes. 
Embora essa riqueza de dados seja interessante, eles são apresentados em uma lista corrida, de maneira que buscas se tornam cansativas. 
Finalmente, ela não é exatamente uma agenda cultural dos eventos da cidade. 

\begin{figure}
    \centering
    \caption[Página da agenda JC]{Página da agenda JC}
    \includegraphics*[width = 1.0\linewidth, height = 0.9\textheight]{figs/capturas-solucoes-existentes/agenda-jcnet.png}\\
    \fonte{\cite{agenda-jcnet}}
    \label{fig:agenda-jcnet}
\end{figure}

Similar à Agenda Social Bauru, a Agenda JaúClick \cite{agenda-jauclick} é uma página da web que reúne atividades culturais da cidade de Jaú, dirigida pela Jauclick. 
A página lista eventos individualmente, por meio de uma pequena imagem (\emph{Thumbnail}), e pode ser conferida na \autoref{fig:agenda-jauclick}. 
Cada evento tem sua própria página com uma imagem promocional, porém não há nenhum link externo para a página do evento. 
Assim como as plataformas bauruenses, esta agenda também é restrita (pois é montada pelos administradores), e é guiada por uma intenção publicitária. 
Assim como as demais, não é possível organizar os eventos por tipo ou por dia, apenas listá-los.

\begin{figure}
    \centering
    \caption[Página da agenda Jaúclick]{Página da agenda Jaúclick}
    \includegraphics*[width = 1.0\linewidth, height = 0.9\textheight]{figs/capturas-solucoes-existentes/agenda-jauclick.png}\\
    \fonte{\cite{agenda-jauclick}}
    \label{fig:agenda-jauclick}
\end{figure}

Há também a plataforma Qualaboa \cite{plataforma-qualaboa}, homônima à desenvolvida neste trabalho.
Ela tem a função de agregar eventos, a partir de diversas fontes, e permitir que o usuário navegue entre as opções.
Contudo, esta plataforma busca eventos apenas em plataformas de venda de ingressos (como Sympla, Shotgun e Ingresse), o que implica num foco estritamente promocional, restrito a um tipo específico de eventos, e não permite que entidades organizadoras cadastrem seus eventos diretamente na plataforma.
É possível visualizar a página principal da plataforma na \autoref{fig:agenda-qualaboa}

\begin{figure}
    \centering
    \caption[Página da plataforma Qualaboa]{Página da plataforma Qualaboa}
    \includegraphics[width = 1.0\linewidth]{figs/capturas-solucoes-existentes/agenda-qualaboa.png}\\
    \fonte{\cite{plataforma-qualaboa}}
    \label{fig:agenda-qualaboa}
\end{figure}

Vale notar que a \autoref{fig:agenda-socialbauru}, a \autoref{fig:agenda-jcnet} e a \autoref{fig:agenda-jauclick} foram editadas, tendo a suas partes inferiores cortadas, com a finalizade de serem exibidas neste trabalho.
O conteúdo cortado consiste na continuação de cada página, onde são listados mais eventos, de maneira similar ao que já é representado nas imagens.

Tendo em mente estes veículos, é possível notar algumas fraquezas. 
Nenhuma delas permite que entidades organizadoras publiquem diretamente seus eventos, tendo seus eventos manualmente selecionados por uma equipe de redação, ou agregado de outras plataformas. 
Em relação à exibição dos eventos, algumas plataformas apresentam os eventos em uma lista corrida, e poucos oferecem ferramentas de busca e filtragem para pesquisar eventos.
Também há um problema em relação aos dados de cada evento, uma vez que se encontram todos os dados relevantes a um evento, como hora, local, imagens ilustrativas ou um link para contato.
Ademais, o caráter publicitário tende a favorecer eventos patrocinados. 
No quadro abaixo, é possível comparar as características de cada uma das soluções existentes com a aplicação desenvolvida neste projeto:

% \textbf{\makecell{Trabalho\\desenvolvido}}
\begin{quadro}
    \caption{Comparativo entre soluções existentes e o trabalho desenvolvido.}
    \begin{tabular}{|c|c|c|c|c|c|}
        \hline
        \textbf{Características}                & \textbf{\makecell{Social\\Bauru}}      & \textbf{\makecell{Agenda\\JC}}         & \textbf{JaúClick}          & \textbf{Qualaboa}          & \textbf{\makecell{Trabalho\\desenvolvido}} \\ 
        \hline
        \makecell{Publicação\\direta\\de evento} & \makecell{Não é\\possível} & \makecell{Não é\\possível} & \makecell{Não é\\possível} & \makecell{Não é\\possível} & É possível \\ 
        \hline
        \makecell{Apresentação\\dos dados}      & Fraca                      & Fraca                      & Ótima                      & Regular                    & Ótima \\ 
        \hline
        \makecell{Ferramentas\\de pesquisa}     & Fracas                     & Fracas                     & Fracas                     & Regular                    & Ótimas \\ 
        \hline
        \makecell{Completude\\dos dados}        & Baixa                      & Baixa                      & Regular                    & Regular                    & Alta \\ 
        \hline
        \makecell{Caráter\\publicitário}        & Sim                        & Sim                        & Sim                        & Sim                        & Não \\ 
        \hline
    \end{tabular}
\fonte{Elaborada pelo autor.}
\label{tabela:comparativo-solucoes-existentes}
\end{quadro}