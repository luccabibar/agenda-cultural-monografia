% --------------------------------------------------------
% DEFINIÇÕES DO DOCUMENTO
% --------------------------------------------------------

\documentclass[
	% -- opções da classe memoir --
	12pt,				% tamanho da fonte
	openright,			% capítulos começam em pág ímpar (insere página vazia caso preciso)
	oneside,			% para impressão em verso e anverso. Oposto a twoside
	a4paper,			% tamanho do papel.
	% -- opções da classe abntex2 --
	%chapter=TITLE,		% títulos de capítulos convertidos em letras maiúsculas
	%section=TITLE,		% títulos de seções convertidos em letras maiúsculas
	%subsection=TITLE,	% títulos de subseções convertidos em letras maiúsculas
	%subsubsection=TITLE,% títulos de subsubseções convertidos em letras maiúsculas
	% -- opções do pacote babel --
	english,			% idioma adicional para hifenização
	french,				% idioma adicional para hifenização
	spanish,			% idioma adicional para hifenização
	brazil,				% o último idioma é o principal do documento
]{lib/abntex2}


% --------------------------------------------------------
% PACOTES
% --------------------------------------------------------
\usepackage{cmap}				% Mapear caracteres especiais no PDF
\usepackage[table,xcdraw]{xcolor}
%\usepackage{lmodern}			% Usa a fonte Latin Modern
\usepackage{helvet}			    % Usa a fonte Helvetica (Arial) 
\usepackage[T1]{fontenc}		% Selecao de codigos de fonte.
\usepackage[utf8]{inputenc}		% Codificacao do documento (conversão automática dos acentos)
\usepackage{lastpage}			% Usado pela Ficha catalográfica
\usepackage{indentfirst}		% Indenta o primeiro parágrafo de cada seção.
\usepackage{xcolor}				% Controle das cores    
\usepackage{graphicx}			% Inclusão de gráficos
\usepackage{lipsum}				% para geração de dummy text
\usepackage{listings}
\usepackage{amsmath}

\let\printglossary\relax
\let\theglossary\relax
\let\endtheglossary\relax
\usepackage{lib/update-abntex}

\usepackage[brazilian,hyperpageref]{}	 % Paginas com as citações na bibl
\usepackage{microtype} 

\usepackage{silence}
%Disable all warnings issued by latex starting with "You have..."
\WarningFilter{latex}{You have requested package}
\usepackage[alf, abnt-etal-list=0]{lib/abntex2cite}	% Citações padrão ABNT
\usepackage[br]{lib/nicealgo}       % Pacote para criação de algoritmos
\usepackage{lib/customizacoes}      % Pacote de customizações do abntex2

\usepackage{listings}
\usepackage[normalem]{ulem} % Strikethrough package

\usepackage[]{pdfpages} % Anexo de PDFS (Ficha Catalográfica)

\usepackage{makecell} % Formatavao de tabelas
% \usepackage{hyperref} % referencia
% \usepackage{biblatex} % bibliografia (citetitle)

% definicoes adicionais para nameref
\newcommand{\lcnameref}[1]{%
\bgroup
\let\nmu\MakeLowercase
\nameref{#1}\egroup}
\newcommand{\fcnameref}[1]{%
\bgroup
\def\nmu{\let\nmu\MakeLowercase}%
\nameref{#1}\egroup}

\newcommand{\nmu}{}


% --------------------------------------------------------
% CONFIGURAÇÕES DE PACOTES
% --------------------------------------------------------

% Configurações do pacote listing
\renewcommand{\lstlistingname}{Código} %Mudança no caption do listing para Código
\renewcommand{\lstlistlistingname}{Lista de códigos} %Mudança no caption da lista de listings.

% Contagem de códigos sem incluir o número do capítulo
\usepackage{chngcntr}
\AtBeginDocument{\counterwithout{lstlisting}{chapter}}


% Configurações do pacote backref
\renewcommand{\familydefault}{\sfdefault}
% Usado sem a opção hyperpageref de backref
% \renewcommand{\backrefpagesname}{Citado na(s) página(s):~}
% Texto padrão antes do número das páginas
% \renewcommand{\backref}{}
% Define os textos da citação
% \renewcommand*{\backrefalt}[4]{
% 	\ifcase #1 %
% 		Nenhuma citação no texto.%
% 	\or
% 		Citado na página #2.%
% 	\else
% 		Citado #1 vezes nas páginas #2.%
% 	\fi}%

% --------------------------------------------------------
% INFORMAÇÕES DO ESTILO DE CÓDIGO FONTE
% --------------------------------------------------------

% Color definitions
\definecolor{codegreen}{rgb}{0,0.6,0}
\definecolor{codegray}{rgb}{0.5,0.5,0.5}
\definecolor{codepurple}{rgb}{0.58,0,0.82}
\definecolor{backcolour}{rgb}{1,1,1}

% Definition of Style of Code
\lstdefinestyle{style_code}{
    backgroundcolor=\color{backcolour},   
    commentstyle=\color{codegreen},
    keywordstyle=\color{magenta},
    numberstyle=\tiny\color{codegray},
    stringstyle=\color{codepurple},
    basicstyle=\ttfamily\footnotesize,
    breakatwhitespace=false,         
    breaklines=true,                 
    captionpos=b,                    
    keepspaces=true,                 
    numbers=left,                    
    numbersep=5pt,                  
    showspaces=false,                
    showstringspaces=false,
    showtabs=false,                  
    tabsize=2
}

\lstset{
    language=Python,
    style=style_code, 
    inputencoding=utf8,
    extendedchars = true,
    literate={á}{{\'a}}1 {é}{{\'e}}1 {í}{{\'i}}1 {ó}{{\'o}}1 {ú}{{\'u}}1
    {Á}{{\'A}}1 {É}{{\'E}}1 {Í}{{\'I}}1 {Ó}{{\'O}}1 {Ú}{{\'U}}1
    {à}{{\`a}}1 {è}{{\`e}}1 {ì}{{\`i}}1 {ò}{{\`o}}1 {ù}{{\`u}}1
    {À}{{\`A}}1 {È}{{\`E}}1 {Ì}{{\`I}}1 {Ò}{{\`O}}1 {Ù}{{\`U}}1
    {ä}{{\"a}}1 {ë}{{\"e}}1 {ï}{{\"i}}1 {ö}{{\"o}}1 {ü}{{\"u}}1
    {Ä}{{\"A}}1 {Ë}{{\"E}}1 {Ï}{{\"I}}1 {Ö}{{\"O}}1 {Ü}{{\"U}}1
    {â}{{\^a}}1 {ê}{{\^e}}1 {î}{{\^i}}1 {ô}{{\^o}}1 {û}{{\^u}}1
    {Â}{{\^A}}1 {Ê}{{\^E}}1 {Î}{{\^I}}1 {Ô}{{\^O}}1 {Û}{{\^U}}1
    {ã}{{\~a}}1 {ẽ}{{\~e}}1 {ĩ}{{\~i}}1 {õ}{{\~o}}1 {ũ}{{\~u}}1
    {Ã}{{\~A}}1 {Ẽ}{{\~E}}1 {Ĩ}{{\~I}}1 {Õ}{{\~O}}1 {Ũ}{{\~U}}1
    {œ}{{\oe}}1 {Œ}{{\OE}}1 {æ}{{\ae}}1 {Æ}{{\AE}}1 {ß}{{\ss}}1
    {ű}{{\H{u}}}1 {Ű}{{\H{U}}}1 {ő}{{\H{o}}}1 {Ő}{{\H{O}}}1
    {ç}{{\c c}}1 {Ç}{{\c C}}1 {ø}{{\o}}1 {å}{{\r a}}1 {Å}{{\r A}}1
    {€}{{\euro}}1 {£}{{\pounds}}1 {«}{{\guillemotleft}}1
    {»}{{\guillemotright}}1 {ñ}{{\~n}}1 {Ñ}{{\~N}}1 {¿}{{?`}}1 {¡}{{!`}}1 
}

% --------------------------------------------------------
% INFORMAÇÕES DE DADOS PARA CAPA E FOLHA DE ROSTO
% --------------------------------------------------------

\titulo{Agenda Cultural Virtual
“Qual a Boa” }
\autor{Lucca Tadeu Silva Bibar}
\local{Bauru}
\data{Novembro/2024}
\orientadora{Profa. Dra. Simone das Graças Domingues Prado}
%\coorientador{Prof. Dr. Seu Coorientador}
\instituicao{%
  Universidade Estadual Paulista ``Júlio de Mesquita Filho''
  \par
  Faculdade de Ciências
  \par
  Sistemas de Informação}
\tipotrabalho{Trabalho de Conclusão de Curso}
\preambulo{Trabalho de Conclusão de Curso Bacharelado em Sistemas de Informação da Universidade Estadual Paulista ``Júlio de Mesquita Filho'', Faculdade de Ciências, Campus Bauru.}


% --------------------------------------------------------
% CONFIGURAÇÕES PARA O PDF FINAL
% --------------------------------------------------------

% alterando o aspecto da cor azul
\definecolor{blue}{RGB}{41,5,195}

% informações do PDF
\makeatletter
\hypersetup{
     	%pagebackref=true,
		pdftitle={\@title},
		pdfauthor={\@author},
    	pdfsubject={\imprimirpreambulo},
	    pdfcreator={LaTeX with abnTeX2},
		pdfkeywords={abnt}{latex}{abntex}{abntex2}{trabalho acadêmico},
		colorlinks=true,       		% false: boxed links; true: colored links
    	linkcolor=black,          	% color of internal links
    	citecolor=black,        	% color of links to bibliography
    	filecolor=magenta,      	% color of file links
		urlcolor=black,
		bookmarksdepth=4
}
\makeatother


% ---
% Posiciona figuras e tabelas no topo da página quando adicionadas sozinhas
% em um página em branco. Ver https://github.com/abntex/abntex2/issues/170
\makeatletter
\setlength{\@fptop}{5pt} % Set distance from top of page to first float
\makeatother
% ---

% ---
% Possibilita criação de Quadros e Lista de quadros.
% Ver https://github.com/abntex/abntex2/issues/176
%
\newcommand{\quadroname}{Quadro}
\newcommand{\listofquadrosname}{Lista de quadros}

\newfloat[chapter]{quadro}{loq}{\quadroname}
\newlistof{listofquadros}{loq}{\listofquadrosname}
\newlistentry{quadro}{loq}{0}

% configurações para atender às regras da ABNT
\setfloatadjustment{quadro}{\centering}
\counterwithout{quadro}{chapter}
\renewcommand{\cftquadroname}{\quadroname\space} 
\renewcommand*{\cftquadroaftersnum}{\hfill--\hfill}

\setfloatlocations{quadro}{hbtp}
% ---

% --------------------------------------------------------
% PACOTES ADICIONAIS (TABELAS E FIGURAS)
% --------------------------------------------------------
\usepackage{float}
\usepackage{multirow}
\usepackage{graphicx}
%ISSO FOI COMENTADO PRA SUPRIMIR UM ERRO, SE DER PROBLEMA SÓ COLOCAR DE VOLTA
%\usepackage[table,xcdraw]{xcolor}

% --------------------------------------------------------
% ESPAÇAMENTOS ENTRE LINHAS E PARÁGRAFOS
% --------------------------------------------------------

% O tamanho do parágrafo é dado por:
\setlength{\parindent}{1.3cm}

% Controle do espaçamento entre um parágrafo e outro:
\setlength{\parskip}{0.2cm}


% --------------------------------------------------------
% COMPILANDO O ÍNDICE
% ---------------------------------------------------
\makeindex
% ---
 
% ---
% GLOSSARIO
% ---
% \makeglossaries
% ---
% Exemplo de configurações do glossairo
% \renewcommand*{\glsseeformat}[3][\seename]{\textit{#1}  
%  \glsseelist{#2}}
% ---
 
% --------------------------------------------------------
% INÍCIO DO DOCUMENTO
% --------------------------------------------------------

\begin{document}

% Seleciona o idioma do documento (conforme pacotes do babel)
\selectlanguage{brazil}

% Retira espaço extra obsoleto entre as frases.
\frenchspacing


% --------------------------------------------------------
% ELEMENTOS PRÉ-TEXTUAIS
% --------------------------------------------------------
%\pageref{LastPage} % Número de Páginas do TCC

% Capa
\imprimircapa

% Folha de rosto
% (o * indica que haverá a ficha bibliográfica)
% \imprimirfolhaderosto*
\imprimirfolhaderosto

% TODO: ficha bibliografica

% Inserir a ficha bibliografica: https://www.biblioteca.unesp.br/ficha/
% \begin{fichacatalografica}
%     \includepdf[pages=1, scale=1.0, frame=false]{ficha_catalografica.pdf}
% \end{fichacatalografica}

% TODO: folha de aprovacao

% Inserir folha de aprovação
% \begin{folhadeaprovacao}
% 	\begin{center}
% 		{\ABNTEXchapterfont\large\imprimirautor}
% 		\vspace*{\fill}\vspace*{\fill}
% 		\begin{center}
% 			\ABNTEXchapterfont\bfseries\Large\imprimirtitulo
% 		\end{center}
% 		\vspace*{\fill}
% 		\hspace{.45\textwidth}
% 		\begin{minipage}{.5\textwidth}
% 			\imprimirpreambulo
% 		\end{minipage}%
% 		\vspace*{\fill}
% 	\end{center}
% 	\center Banca Examinadora \\
%     \vspace{\fill}
% 	\textbf{\imprimirorientador} \\ Orientadora\\
% 		Universidade Estadual Paulista "Júlio de Mesquita Filho"\\
% 		Faculdade de Ciências \\
% 	Departamento de Ciência da Computação \\
%     \vspace{\fill}
% 	\textbf{Prof. AAAA BBB CCC} \\
% 		Universidade Estadual Paulista "Júlio de Mesquita Filho"\\
% 		Faculdade de Ciências \\
% 	Departamento de Ciência da Computação \\
%     \vspace{\fill}
% 	\textbf{Prof. AAAA BBB CCC} \\
% 		Universidade Estadual Paulista "Júlio de Mesquita Filho"\\
% 		Faculdade de Ciências \\
% 	Departamento de Ciência da Computação \\
%     \vspace{\fill}
% 	\begin{center}
% 		\vspace*{0.5cm}
% 		\par
% 		{Bauru, 14 de Novembro de 2024.} 
% 		\vspace*{1cm}
% 	\end{center}
% \end{folhadeaprovacao}

% TODO: Dedicatoria

% Dedicatória
% \begin{dedicatoria}
% 	\vspace*{\fill}
% 	\begin{flushright}
% 		\textit{Este trabalho é dedicado à minha família que me incentivou a vir até aqui.} 
% 	\end{flushright}
% \end{dedicatoria}

% TODO: Agradecimentos

% Agradecimentos
% \begin{agradecimentos}

% Agradeço a Deus, à minha família e a todos os meus amigos que me acompanharam nessa jornada. Pai e mãe, obrigado sobretudo a vocês, que são a razão da minha vida. Apesar de todos os problemas e desafios, nunca deixaram de me amar e apoiar. A cada dia longe de vocês, aprendi a valorizar ainda mais a importância de tê-los juntos e ao meu lado. Em especial, gostaria de agradecer aos meus queridos padrinhos, Luís Alfredo e Maria Cecília, e à minha tia Dirce, que se fizeram presentes em minha vida durante todos os momentos. Vocês incentivaram esta jornada e hoje repousam ao lado de Deus. Agradeço também a minha avó, todos os meus tios e tias, meus irmãos e meus primos. 

% Quero agradecer a todos que ajudaram na minha formação como humano e cidadão, e me incentivaram a crescer. Obrigado aos meus amigos, que aqui fiz e aqueles que vieram comigo. Devo muito por todo o apoio, carinho, atenção e por enfrentarem todos os desafios ao meu lado até aqui, e por estarem prontos para os que virão. Agradeço especialmente ao professor Sementille, o melhor orientador que poderia ter, que entendeu meus problemas e medos e sempre ofereceu suporte durante toda a graduação. Juntamente, obrigado Davi, sua dedicação e generosidade fizeram toda a diferença neste trabalho.

% \end{agradecimentos}

% TODO: epigrafe

% Epígrafe
% \begin{epigrafe}
% 	\vspace*{\fill}
% 	\begin{flushright}
%             \textit{"People won't know how you feel unless you tell them."}\\
% 		- Frieren.
% 	\end{flushright}
% \end{epigrafe}


% --------------------------------------------------------
% RESUMOS
% --------------------------------------------------------

% resumo em português
\setlength{\absparsep}{18pt} % ajusta o espaçamento dos parágrafos do resumo
\begin{resumo}
	A cultura e o lazer são partes fundamentais da experiência humana, especialmente em sociedade. Portanto, o limitado tempo disponível para atividades culturais (sejam elas shows, peças, eventos ou festas) deve ser bem investido. Entretanto, ao decidir participar de alguma atividade cultural, nem sempre está claro quais são as opções disponíveis numa determinada data ou região, visto que as informações de quais eventos ocorrem estão dispostas de maneira dispersa em meios que nem sequer foram projetados para hospedar este tipo de informação, como em perfis em redes sociais, blogs ou jornais locais. O presente trabalho visa então apresentar uma alternativa: uma plataforma que reúna todas as atividades culturais de uma determinada região, apresentando-as de maneira concisa e organizada, sem que seja necessário demasiado esforço para conferi-las.\\
	\textbf{Palavras-chave:} Cultura, Lazer, Angular, Spring, Cliente-Servidor.
\end{resumo}

% TODO: plavras chave; Abstract

% resumo em inglês
\begin{resumo}[Abstract]
	\begin{otherlanguage*}{english}
		Culture and Leisure are fundamental aspects of the human experience, especially in society. Therefore, the limited time that is available for cultural activities (such as concerts, plays, or parties) must be well spent. However, when deciding to partake in a cultural activity, it may not be clear what are the available options in a given date or region, as the information of what events are happening are laid out in media that are not even designed to host this kind of information, such as in social media profiles, blogs, or local newspapers. This project then aims to present an alternative: a platform that gathers all cultural activities in a given region, presenting them in a concise and organized way, without the need of unnecessary effort to check on them.\\
		\textbf{Keywords:} Culture, Leisure, Angular, Spring, Client-Server.
	\end{otherlanguage*}
\end{resumo}


% --------------------------------------------------------
% LISTA DE ILUSTRAÇÕES
% --------------------------------------------------------

% inserir lista de ilustrações
\pdfbookmark[0]{\listfigurename}{lof}
\listoffigures*
\cleardoublepage

% --------------------------------------------------------
% LISTA DE QUADROS
% --------------------------------------------------------
\pdfbookmark[0]{\listofquadrosname}{loq}
\listofquadros*
\cleardoublepage
% ---

% --------------------------------------------------------
% LISTA DE TABELAS
% --------------------------------------------------------

% inserir lista de tabelas
%\pdfbookmark[0]{\listtablename}{lot}
%\listoftables*
%\cleardoublepage

% --------------------------------------------------------
% LISTA DE ABREVIATURAS E SIGLAS
% --------------------------------------------------------

% TODO: Siglas

% \begin{siglas}
%     \item[2D] Duas Dimensões 
%     \item[3D] Três Dimensões 
%     \item[AM] Aprendizado de Máquina 
%     \item[ASL] \emph{American Sign Language}
%     \item[AVI] \emph{Audio Video Interleave}
%     \item[CSL] \emph{Chinese Sign Language}
%     \item[CNN] \emph{Convolutional Neural Network}
%     \item[HD] \emph{High Definition}
%     \item[IA] Inteligência Artificial
%     \item[IBGE] Instituto Brasileiro de Geografia e Estatística
%     \item[JSON] \emph{JavaScript Object Notation} 
%     \item[KNN] \emph{K-Nearest Neighbors}
%     \item[LIBRAS] Língua Brasileira de Sinais 
%     \item[MP4] \emph{Moving Picture Experts Group 4}
%     \item[OMS] Organização Mundial da Saúde 
%     \item[RF] \emph{Random Forest}
%     \item[RGB] \emph{Red, Green and Blue}
%     \item[RGBD] \emph{Red, Green, Blue and Depth}
%     \item[SLERP] \emph{Spherical Linear Interpolation}

% \end{siglas}
% --------------------------------------------------------

% --------------------------------------------------------
% SUMÁRIO
% --------------------------------------------------------

% inserir o sumario
\pdfbookmark[0]{\contentsname}{toc}
\tableofcontents*
\cleardoublepage


% --------------------------------------------------------
% ELEMENTOS TEXTUAIS
% --------------------------------------------------------

\pagestyle{simple}

% Arquivos .tex do texto, podendo ser escritos em um único arquivo ou divididos da forma desejada

\chapter{Introdução}
\label{ch.introducao}

A cultura, lazer, e experiências em comunidade são partes importantíssimas da experiência humana, aspectos definitivos da nossa espécie e sociedade, ao ponto de serem estabelecidos como direitos de todo ser humano, na Declaração Universal dos Direitos Humanos \citeonline{direitos-humanos}. Entretanto, atividades que fomentam estes aspectos se tornam cada vez mais inacessíveis em função de uma sociedade onde o tempo é um recurso escasso que deve ser bem investido, e que caso contrário, será gasto navegando por redes sociais sem agregar nada. De acordo com \citeonline{acesso-cultura}, existem diversos fatores que contribuem para a alienação da população da cultura, especialmente em demográficos de baixa renda.

Um desses fatores que dificulta o acesso à essas atividades é justamente a desorganização na maneira como elas são divulgadas. A grande maioria das entidades organizadoras que gerenciam eventos culturais (sejam shows, peças, festas universitárias, e afins) utilizam o Instagram como principal veículo de divulgação. Isso se torna um problema devido ao fato que cada entidade organizadora tem seu próprio perfil: se um usuário não conhece uma determinada organização, dificilmente ficará sabendo dos eventos que ela promove por meio  do instagram. Também é difícil explorar as opções de eventos, visto que a busca não é eficiente, não leva em consideração a data, e compete com demais tipos de postagem.

% Um desses fatores que dificulta o acesso à essas atividades é justamente a desorganização na maneira como elas são divulgadas. A grande maioria das organizações que gerenciam eventos culturais dos mais variados tipos (sejam shows, peças, festas universitárias, e afins) utiliza o Instagram como principal veículo de divulgação. Isso se torna um problema devido ao fato que cada organização tem seu próprio perfil: se um usuário não conhece uma determinada organização, dificilmente ficará sabendo dos eventos que ela organiza por meio do instagram. Também é difícil explorar as opções de eventos, visto que a busca não é eficiente, e não leva em consideração a data, e compete com demais tipos de postagem

Com isso em foco, é possível reconhecer o atrito que existe no processo de buscar, explorar e conhecer eventos culturais, bem como no próprio processo de divulgação deles. O presente projeto se propõe a prover um remédio para esta situação: uma plataforma de divulgação de eventos, que sirva como uma agenda cultural. Nela, usuários poderão navegar pelos eventos que ocorrem em sua região, podendo buscar ou filtrá-los, permitindo assim que se escolha uma atividade cultural que mais lhe agrada, valorizando o seu tempo de lazer. as entidades responsáveis por atividades culturais poderão divulgar seus eventos numa plataforma dedicada, assim facilitando o encontro com a sua demografia alvo. Desta forma, os usuários vão em eventos que gostam, e as organizações encontram o seu público! 

O principal obstáculo para a implementação de um sistema com esses objetivos é justamente romper o \textit{status quo}, visto que todos os potenciais usuários já estão bem acostumados com como ocorre a divulgação de eventos, mesmo que ela não corra de maneira ideal. Como tanto os usuários quanto as entidades organizadoras são os dois atores fundamentais deste sistema, é imprescindível que se incentive a adesão deles à plataforma, seja por meio de divulgação, ou oferecendo o mínimo de dificuldade de acesso à plataforma. Uma vez que as pessoas conhecerem um sistema de qualidade, que oferece os dados e funcionalidades que eles precisam, a expectativa é de que tornem este o veículo principal para a divulgação e consulta deste tipo e conteúdo.

\section{Detalhamento do Problema}
\label{sec.detalhamento-problema}
    
Como descrito início do capítulo, o principal problema identificado é  o atrito desnecessário que ocorre quando uma pessoa tenta buscar uma atividade ou evento para participar. A grande maioria das entidades que gerenciam eventos culturais (sejam shows, festas universitárias, eventos esportivos, oficinas, e afins) utilizam o Instagram como principal (e, às vezes, único) veículo de comunicação. Como foi pontuado, o Instagram não oferece ferramentas de busca, filtragem ou pesquisa adequadas para agir como uma agenda cultural: um usuário dificilmente descobrirá novos eventos além dos que já conhece, a menos que se lhe seja recomendado uma postagem, sistema esse que favorece fortemente publicações patrocinadas.

Isso porque o Instagram sequer é projetado para isso. Ele é uma rede social feita para o compartilhamento de fotos por pessoas, que foi evoluindo e se adaptando, incorporando diferentes tipos de conteúdo, incluindo divulgação de eventos. Dito isso, ele não é a plataforma ideal para hospedar este tipo de conteúdo. Há ainda organizações que se utilizam de grupos e lista de divulgação no WhatsApp que, embora convenientes, também não estão no ambiente ideal e compartilham problemas similares.

Existem ainda plataformas de mídias locais, como o Social Bauru ou JauClick, que visam justamente agregar este tipo de conteúdo, e são bem sucedidas nesta tarefa. Entretanto, esse trabalho ainda tem pontos de melhorias, como a democratização à divulgação de conteúdo, ou o uso de uma plataforma centralizada e especializada para isto (ao invés de depender de diversas plataformas locais).

Resolver essas dificuldades acumuladas pode incentivar mais pessoas a frequentarem eventos culturias, o que seria benéfico tanto para os participantes dos eventos, que teriam experiências de qualidade, quanto para os próprios organizadores de eventos, que conseguiria alcançar uma porção maior de seu público alvo. Cabe ressaltar que isso favoreceria as organizações menores em especial, como casas de shows locais, que têm que lidar não só com meios inadequados, mas também com a concorrência de organizações maiores e mais famosas.

% Toda essas dificuldades se acumulam o que resulta numa taxa  reduzida de participação em eventos, o que é prejudicial tanto para os potenciais participantes de eventos, que terminam com menos experiências, quanto para os próprios organizadores de eventos, que acabam com um público reduzido simplesmente porque o demográfico em potencial não conhece o evento. Em especial, cabe ressaltar que isso é um problema maior ainda para as organizações menores, como casas de shows locais, que têm que lidar não só com meios inadequados, mas também com a concorrência de organizações maiores e mais famosas.

% Diminuir essas dificuldades seria benéfico para tanto frequentadores quanto organizações de eventos, pois aumentar a taxa de participação nos eventos é amplificar uma troca mútua e benéfica para ambas as partes.

\section{Objetivos}
\label{sec.objetivos}

\subsection{Objetivo Geral}
\label{ssc.objetivo_geral}

O presente trabalho tomou como alvo desenvolver uma aplicação que aja como uma agenda cultural: onde usuários podem conferir os eventos culturais que ocorrem na sua região, e os organizadores desses eventos possam os publicar a fim de divulgá-los ao seu público.
A aplicação desenvolvida teve como princípio norteador a facilidade de uso, priorizando oferecer uma experiência amigável, mas sem sacrificar a funcionalidade.
A aplicação desenvolvida teve como princípios o uso simplificado, funcionabilidade robusta e acesso democrático.

\subsection{Objetivos Específicos}
\label{ssc.objetivos_especificos}

É possível nomear os seguintes objetivos específicos para o desenvolvimento da aplicação.
Note que alguns dos itens dialogam com as outras soluções existentes, que serão discutidos no \label{ch.solucoes-existentes}.

\begin{itemize}
    \item Agregar eventos culturais, e exibi-los de maneira clara e organizada.
    \item Permitir que qualquer entidade organizadora possa divulgar seus eventos. 
    \item Minimizar o atrito de uso.
    \item Oferecer mecanismos de busca e filtragem robustos.
    \item Suprir demais funcionalidades já providas por outras soluções.
\end{itemize}    

\section{Organização da Monografia}
\label{sec.organizacao-monografia}

Esta monografia vai discorrer sobre o problema apresentado, bem como a proposta do projeto que visa resolvê-lo, assumindo a seguintes estrutura:

% Esta \nameref{ch.introducao}, que apresenta a ideia central, contém o \nameref{sec.detalhamento-problema}, que aprofunda a discussão sobre a problemática apresentada. 

% O capítulo de \nameref{ch.solucoes-existentes} discorre sobre as opções existentes que já tentam resolver este problema, expondo características que norteiam os objetivos deste projeto, descritos logo em seguida na seção \nameref{sec.objetivos}. 

% O capítulo \nameref{ch.descricao-projeto} explica como o sistema vai funcionar e se estruturar, e as seções \nameref{sec.tecnologias-utilizadas} e \nameref{sec.procedimentos-validacao} discutem quais ferramentas serão utilizadas em sua implementação, e como os módulos deste sistema serão testados e validados, respectivamente.

% O capítulo \nameref{ch.analise-riscos} descreve quais riscos que o projeto corre, bem como uma abordagem para preveni-los e remediá-los. O \nameref{ch.cronograma} informa como o desenvolvimento do projeto ocorrerá ao longo do tempo, situando-o de acordo com a disciplina de TCC 1. E Por fim, a \nameref{ch.conclusao} resume os principais pontos abordados por este trabalho.

% sucumba nameref

Este \autoref{ch.introducao} apresenta a ideia central, e contém a \autoref{sec.detalhamento-problema}, que aprofunda a discussão sobre a problemática apresentada. O \autoref{ch.solucoes-existentes} discorre sobre as opções existentes que já tentam resolver este problema, expondo características que norteiam os objetivos deste projeto, descritos logo em seguida na \autoref{sec.objetivos}. 

O \autoref{ch.descricao-projeto} explica como o sistema vai funcionar e se estruturar, e a \autoref{sec.tecnologias-utilizadas} e \autoref{sec.procedimentos-validacao} discutem quais ferramentas serão utilizadas em sua implementação, e como os módulos deste sistema serão testados e validados, respectivamente. Por fim, o \autoref{ch.conclusao} resume os principais pontos abordados por este trabalho. % TODO: Atualizar quando escrever demais capitulos

% O \autoref{ch.analise-riscos} descreve quais riscos que o projeto corre, bem como uma abordagem para preveni-los e remediá-los. O \autoref{ch.cronograma} informa como o desenvolvimento do projeto ocorrerá ao longo do tempo, situando-o de acordo com a disciplina de TCC 1. E Por fim, o \autoref{ch.conclusao} resume os principais pontos abordados por este trabalho.

\chapter{Soluções Existentes}
\label{ch.solucoes-existentes}

Já existem alguns outros meios que tentam agregar as atividades culturais da região de Bauru. 
Entretanto, nenhuma delas apresenta o conjunto de características com a completude que o presente trabalho visa implementar. 
Seguem exemplos:

A Agenda Social Bauru \cite{agenda-socialbauru} é uma página da web que reúne diversos programas culturais da cidade, dirigida pela Social Bauru, uma mídia online focada em divulgar atividades e atrações. 
Conforme representada na \autoref{fig:agenda-socialbauru}, a página contém diversos eventos, organizados por categorias e separados por dia, ao longo de uma lista em texto corrido. 
Entretanto, a coluna é restrita (visto que é escrita por um redator da agência, não cumprindo então a função de uma plataforma aberta), e é orientada, sobretudo, por uma intenção publicitária. 
Além disso, é possível visualizar o conteúdo apenas em forma de lista.

\begin{figure}
    \centering
    \caption[Página da agenda Social Bauru]{Página da agenda Social Bauru}
    \includegraphics*[width = 1.0\linewidth, height = 0.9\textheight]{figs/capturas-solucoes-existentes/agenda-socialbauru.png}\\
    \fonte{\cite{agenda-socialbauru}}
    \label{fig:agenda-socialbauru}
\end{figure}

Outra alternativa é a Agenda JC \cite{agenda-jcnet}, da JCNet. 
Ela é mais completa, mas ainda é restrita, por ser escrita por uma redação. 
A página, exibida na \autoref{fig:agenda-jcnet}, lista o horário de funcionamento de diversos estabelecimentos da cidade, como restaurantes, bares e lanchonetes. 
Embora essa riqueza de dados seja interessante, eles são apresentados em uma lista corrida, de maneira que buscas se tornam cansativas. 
Finalmente, ela não é exatamente uma agenda cultural dos eventos da cidade. 

\begin{figure}
    \centering
    \caption[Página da agenda JC]{Página da agenda JC}
    \includegraphics*[width = 1.0\linewidth, height = 0.9\textheight]{figs/capturas-solucoes-existentes/agenda-jcnet.png}\\
    \fonte{\cite{agenda-jcnet}}
    \label{fig:agenda-jcnet}
\end{figure}

Similar à Agenda Social Bauru, a Agenda JaúClick \cite{agenda-jauclick} é uma página da web que reúne atividades culturais da cidade de Jaú, dirigida pela Jauclick. 
A página lista eventos individualmente, por meio de uma pequena imagem (\emph{Thumbnail}), e pode ser conferida na \autoref{fig:agenda-jauclick}. 
Cada evento tem sua própria página com uma imagem promocional, porém não há nenhum link externo para a página do evento. 
Assim como as plataformas bauruenses, esta agenda também é restrita (pois é montada pelos administradores), e é guiada por uma intenção publicitária. 
Assim como as demais, não é possível organizar os eventos por tipo ou por dia, apenas listá-los.

\begin{figure}
    \centering
    \caption[Página da agenda Jaúclick]{Página da agenda Jaúclick}
    \includegraphics*[width = 1.0\linewidth, height = 0.9\textheight]{figs/capturas-solucoes-existentes/agenda-jauclick.png}\\
    \fonte{\cite{agenda-jauclick}}
    \label{fig:agenda-jauclick}
\end{figure}

Há também a plataforma Qualaboa \cite{plataforma-qualaboa}, homônima à desenvolvida neste trabalho.
Ela tem a função de agregar eventos, a partir de diversas fontes, e permitir que o usuário navegue entre as opções.
Contudo, esta plataforma busca eventos apenas em plataformas de venda de ingressos (como Sympla, Shotgun e Ingresse), o que implica num foco estritamente promocional, restrito a um tipo específico de eventos, e não permite que entidades organizadoras cadastrem seus eventos diretamente na plataforma.
É possível visualizar a página principal da plataforma na \autoref{fig:agenda-qualaboa}

\begin{figure}
    \centering
    \caption[Página da plataforma Qualaboa]{Página da plataforma Qualaboa}
    \includegraphics[width = 1.0\linewidth]{figs/capturas-solucoes-existentes/agenda-qualaboa.png}\\
    \fonte{\cite{plataforma-qualaboa}}
    \label{fig:agenda-qualaboa}
\end{figure}

Vale notar que a \autoref{fig:agenda-socialbauru}, a \autoref{fig:agenda-jcnet} e a \autoref{fig:agenda-jauclick} foram editadas, tendo a suas partes inferiores cortadas, com a finalizade de serem exibidas neste trabalho.
O conteúdo cortado consiste na continuação de cada página, onde são listados mais eventos, de maneira similar ao que já é representado nas imagens.

Tendo em mente estes veículos, é possível notar algumas fraquezas. 
Nenhuma delas permite que entidades organizadoras publiquem diretamente seus eventos, tendo seus eventos manualmente selecionados por uma equipe de redação, ou agregado de outras plataformas. 
Em relação à exibição dos eventos, algumas plataformas apresentam os eventos em uma lista corrida, e poucos oferecem ferramentas de busca e filtragem para pesquisar eventos.
Também há um problema em relação aos dados de cada evento, uma vez que se encontram todos os dados relevantes a um evento, como hora, local, imagens ilustrativas ou um link para contato.
Ademais, o caráter publicitário tende a favorecer eventos patrocinados. 
No quadro abaixo, é possível comparar as características de cada uma das soluções existentes com a aplicação desenvolvida neste projeto:

% \textbf{\makecell{Trabalho\\desenvolvido}}
\begin{quadro}
    \caption{Comparativo entre soluções existentes e o trabalho desenvolvido.}
    \begin{tabular}{|c|c|c|c|c|c|}
        \hline
        \textbf{Características}                & \textbf{\makecell{Social\\Bauru}}      & \textbf{\makecell{Agenda\\JC}}         & \textbf{JaúClick}          & \textbf{Qualaboa}          & \textbf{\makecell{Trabalho\\desenvolvido}} \\ 
        \hline
        \makecell{Publicação\\direta\\de evento} & \makecell{Não é\\possível} & \makecell{Não é\\possível} & \makecell{Não é\\possível} & \makecell{Não é\\possível} & É possível \\ 
        \hline
        \makecell{Apresentação\\dos dados}      & Fraca                      & Fraca                      & Ótima                      & Regular                    & Ótima \\ 
        \hline
        \makecell{Ferramentas\\de pesquisa}     & Fracas                     & Fracas                     & Fracas                     & Regular                    & Ótimas \\ 
        \hline
        \makecell{Completude\\dos dados}        & Baixa                      & Baixa                      & Regular                    & Regular                    & Alta \\ 
        \hline
        \makecell{Caráter\\publicitário}        & Sim                        & Sim                        & Sim                        & Sim                        & Não \\ 
        \hline
    \end{tabular}
\fonte{Elaborada pelo autor.}
\label{tabela:comparativo-solucoes-existentes}
\end{quadro}
\input{chapters/03-EspecificacaoProjeto}
\chapter{Tecnologias Utilizadas}
\label{ch.tecnologias-utilizadas}
    
Dado a natureza do projeto, seus objetivos e seus requisitos, é possível definir como ele será implementado, assim como as tecnologias escolhidas para tal. 

Em termos de arquitetura, a mais adequada é a Cliente-Servidor: ela permite que um \emph{Backend} (denominado Servidor) receba, armazene, processe, gerencie e sirva dados, de acordo com pedidos realizados por um \emph{Frontend} (denominado Cliente), através de requisições HTTP feitas através da internet. 
Esta arquitetura é vantajosa pois desacopla o \emph{Frontend} do \emph{Backend}, permitindo que sejam desenvolvidos em paralelo, com dependências reduzidas. 
O único co-requisito entre as duas partes é que elas sejam capazes de se comunicar por meio de requisições HTTP, o que não é uma funcionalidade incomum. 
Adicionalmente, ela permite que diversos Clientes diferentes se comuniquem com o mesmo Servidor, flexibilizando a natureza do Cliente, que pode ser desde uma aplicação web, até um aplicativo mobile.

A fim de garantir a segurança no credenciamento dos usuários, foi adotada uma estratégia de autenticação baseada em \emph{JSON Web Tokens}, ou JWTs.
Para realizar o login, o usuário deve enviar suas credenciais (e-mail e senha, por exemplo) para o servidor.
Uma vez que os dados são validados, o servidor os codifica em um Token, utilizando uma chave privada, e envia o envia de volta para o cliente.
Desta forma, o cliente pode acessar endpoints protegidos ao enviar o Token junto à requisição, para se identificar.

Quando o servidor recebe uma requisição em um endpoint protegido, ele verifica a presença do Token de autenticação, e então o decodifica, assim acessando os dados originalmente armazenados.
Desta forma, o servidor possui uma maneira segura de prover um meio de autenticação ao cliente, uma vez que apenas ele é capaz de codificar e decodificar o Token.
Note que o cliente em nenhum momento precisa se preocupar em processar o Token, uma vez que ele age apenas como uma chave de autenticação para ele.

Para o \emph{Backend} do projeto, foi empregado a linguagem de programação Java, junto ao \emph{framework} Springboot. 
Ele permitiu o desenvolvimento, de maneira robusta e confiável, de um meio para os Clientes acessarem as funcionalidades do Servidor, de maneira padronizada por uma API em estilo REST.
O \emph{framework} oferece amplo apoio para aplicações estilo REST, como controladores REST, filtros de requisições HTTP, e acesso ao banco de dados via JDBC. Utilizando o filtro de requisições, foi possível implementar uma boa solução de autenticação de usuários via JWT, que foi a abordagem escolhida para esta tarefa. 

Para armazenar os dados de usuários de dos eventos, foi escolhido um banco de dados relacional tradicional, o PostgreSQL, devido a sua fácil utilização tanto com o Springboot quanto com a própria arquitetura Cliente-Servidor.

Para o \emph{Frontend}, o Cliente foi implementado como uma aplicação Web, a princípio. Este tipo de aplicação funciona bem na arquitetura Cliente-Servidor, visto que a própria natureza dela implica numa ênfase em requisições HTTP. Utilizando o \emph{framework} Angular, que emprega a linguagem TypeScript, é possível desenvolver uma aplicação robusta para navegadores web. Adicionalmente, aplicações utilizando Angular costumam se comportar bem em navegadores para aparelhos celulares, tornando-se assim uma escolha flexível.

Ambas as linguagens escolhidas são fortemente tipadas, e oferecem apoio à abordagem de programação orientada à objeto. Isto garante a plena comunicação entre cada componente da aplicação, e garante que ela se comporte exatamente como esperado, facilitando o desenvolvimento do projeto.

Deve-se notar que boa parte do público em potencial do serviço utilizaria aparelhos celulares para acessá-lo. Com isso em mente, também é possível desenvolver um aplicativo de celular que agisse como um Cliente. Soluções PWAs, ou \emph{Progressive Web Apps}, são as ideais, visto que permitem desenvolver aplicativos mobile utilizando ferramentas usadas no desenvolvimento web tradicional. No caso do Ionic, um \emph{framework} PWA, é oferecido suporte ao Angular, assim justificando ainda mais a escolha deste, visto que isto tornaria as aplicações web e mobile semelhantes, e reduziria o atrito no desenvolvimento paralelo das duas.
    
% TODO MOVER / REMOVE

% \section{Procedimentos de Validação}
% \label{sec.procedimentos-validacao}

% % TODO: TESTES
    
% Para garantir o funcionamento do sistema, ele será testado durante e após o seu desenvolvimento, empregando a estratégia da Pirâmide de Testes, onde cada nível da pirâmide corresponde a um tipo de teste a ser executado na aplicação. Com essa estrutura, é possível testar a aplicação de maneira integral, garantindo que tudo esteja funcionando em todos os níveis possíveis.

% A base da pirâmide são os Testes Unitários, onde cada componente da aplicação é testado de maneira individual, verificando se o seu comportamento, em isolação, é o esperado. São os testes mais baratos de serem implementados e garantem o funcionamento individual de cada peça do sistema. Eles podem ser implementados a nível de componentes, classes ou funções.

% O meio da pirâmide corresponde aos Testes de Integração, que irão averiguar como os componentes, já testados pelos Testes Unitários, se comunicam entre si. É nesta etapa que são testadas conexões e integrações entre serviços, clientes e servidores e diferentes blocos da aplicação, bem como o funcionamento de páginas como um todo. Portanto, são realizados a nível de páginas, ou de  \emph{endpoints} completos.

% O topo desta pirâmide diz respeito aos Testes Ponta-a-Ponta (ou End to End, E2E), que são os mais complexos. Eles testam a aplicação seguindo uma abordagem de Caixa-Preta, ou seja, desconsiderando o funcionamento mecânico da aplicação e verificando apenas o sistema como ele é apresentado para o usuário. Em outras palavras, é um teste que envolve utilizar a aplicação tal qual ela foi projetada para, em busca de erros, bugs e inconsistências. Devido à sua natureza, testes E2E são realizados em nível de aplicação.
   
% \section{Produto Desenvolvido}
% \label{sec.produto-desenvolvido}

% Durante o período em que ocorreram as disciplinas TCC 1 e 2, foi desenvolvido um Produto Mínimo Viável, e totalmente funcional, desta aplicação. Uma plataforma que permite às entidades organizadoras de eventos os divulgarem, e, aos usuários, encontrarem eventos relevantes para participarem.  

% O componente \emph{backend} desenvolvido conta com diversos \emph{endpoints}, por onde é possível acessar as funcionalidades do Servidor. 
% Os dois principais grupos de \emph{endpoints} são relacionados aos Eventos, e aos Usuários.
% Por meio deles, é possível tomar todas as ações da aplicação: 
% Criar uma conta e fazer login se referem aos \emph{endpoints} de Usuarios. Já buscar e consumir eventos, bem como criar, editar, moderar e excluí-los, são funcionalidades dos \emph{endpoints} de eventos.
% O compoente backend também conta com um sistema de autenticação via JWT, garantindo que apenas os usuários adequados tem acesso à determinados recursos protegidos.

% O componente \emph{frontend} desenvolvido age como o Cliente que acessa os dados do Servidor, implementando páginas na web que utilizam suas funcionalidades. 
% Ele acessa essas funcionalidades por meio de requisições HTTP aos \emph{endpoints} do Servidor, por meio de um serviço que oferece funcionalidades de HTTP.
% Como usuário, é possível procurar eventos por meio de filtros de busca, e ver seus detalhes além de conseguir visualizar todos os eventos que ocorrem em um determinado dia ou semana, por meio de uma agenda. 
% Como um Organizador, é possível criar, editar e excluir eventos, e como um Moderador, é possível analisar e moderar os eventos.
% É possível se identificar por meio de um sistema de login, que mantém o usuário ativo por meio de \emph{cookies} de navegador.

% A identidade visual da adotada, embora não seja a final, reflete a visão da plataforma, visando ser sóbria, mas descontraída. Assim, ela emprega  um design claro e colorido. 

% % TODO: figuras

% As seguintes figuras, \autoref{fig:captura-busca} e \autoref{fig:captura-evento}, são capturas das respectivas telas, mencionadas no parágrafo acima.

% \begin{figure}[H]
%     \centering
%     \caption[Captura - Tela de Busca]{Captura - Tela de Busca}
%     \includegraphics[width = 1.0\linewidth]{figs/03-descricao-projeto/captura-busca.png}\\
%     \fonte{Elaborada pelo autor.}
%     \label{fig:captura-busca}
% \end{figure}

% \begin{figure}[H]
%     \centering
%     \caption[Captura - Tela de Evento]{Captura - Tela de Evento}
%     \includegraphics[width = 1.0\linewidth]{figs/03-descricao-projeto/captura-evento.png}\\
%     \fonte{Elaborada pelo autor.}
%     \label{fig:captura-evento}
% \end{figure}
% \chapter{Análise de Riscos}
\label{ch.analise-riscos}

Uma vez que se entende a estrutura do sistema, tal qual sua estratégia de negócios, é possível apontar alguns riscos que podem ocorrer durante seu desenvolvimento e atuação.

O primeiro, e mais notável, é a falta de adesão pelos organizadores de eventos, que tem probabilidade moderada, e impacto moderado. Como o conteúdo da plataforma é gerado pelos organizadores, e não por uma redação dedicada, a ausência de publicações referentes à eventos deixaria o conteúdo da plataforma incompleto. Uma variante crítica desde risco, onde quase nenhuma organização utiliza a plataforma, tem uma probabilidade reduzida, mas um impacto catastrófico. 

Para remediar este risco, são adotadas ações: a primeira é entrar em contato com organizadores de eventos locais e convidá-los a utilizar a plataforma, para garantir que estejam cientes de sua existência e potenciais benefícios. A segunda é garantir que publicar na plataforma envolva o menor atrito possível, exigindo o mínimo de esforço do usuário que publicaria o evento. E a terceira, seria publicar manualmente sobre eventos que não estão presentes na plataforma, a fim de suprir uma falta crítica de conteúdo, adotando esta postura paliativamente, de maneira temporária.

Outra possibilidade é a de que a renda da plataforma não a tornarem auto-sustentável, um risco de probabilidade baixa, de alto impacto. A monetização da plataforma depende de anúncios, o que significa que ela depende do uso e adesão de usuários. Entretanto, isto é balanceado pelo baixo custo de manutenção inicial da plataforma, uma vez que não seria necessário uma hospedagem muito custosa devido ao fato que seu escopo inicial se limita à população de Bauru e região.

Para remediar esta possibilidade, é possível investir em marketing e divulgação da plataforma para garantir que usuários em potencial estejam cientes de sua existência e a experimentem. Zelar qualidade da plataforma e seu conteúdo é imprescindível, pois são estes dois fatores que garantem a adesão do usuário: se a plataforma for de fácil utilização e oferecer dados relevantes, os usuários a utilizarão regularmente. Outra opção seria estudar maneiras alternativas de monetização, como publicações patrocinadas.

Vale notar que esses dois riscos estão intimamente ligados: uma plataforma que não tem os eventos não manterá seus usuários pois estará incompleta, e uma plataforma que não tem usuários não encoraja as organizações a publicarem seus eventos nela. Tal qual um motor de carro que utiliza um arranque para começar a funcionar, medidas paliativas (como realizar publicações manuais e campanhas de divulgação) podem ser empregadas nos estágios iniciais do lançamento da plataforma ao público, a fim de garantir que nenhum desses riscos se desenvolva. Uma vez que a plataforma se apresentar estável, essas medidas podem ser abandonadas.

% \input{chapters/05-Cronograma}
\chapter{Resultados}
\label{ch.resultados}

Durante o ciclo de desenvolvimento deste trabalho, foi desenvolvido a aplicação descrita nas seções anteriores, que esta seção seção abordará.

Conforme especificado, a aplicação é composta de duas partes, o \emph{Backend}, que age como o Servidor, e o \emph{Frontend}, que age como o Cliente. 
Em um cenário normal, um servidor executa a porção \emph{Backend} da aplicação, e serve o \emph{Frontend} ao usuário, que o executa em seu dispositivo.
Utilizando um navegador web, é possível acessar a aplicação desta maneira.

% TODO IBAGENS
% TODO TAMBEM: AFERIR EXISTENCIA DA HOMEPAGE
A página principal é a primeira página que um novo usuário vê, por padrão. 
É possível identificar um cabecalho na página, que contém alguns \emph{links} para navegação, e uma barra de pesquisa.
Ao buscar utilizando a barra de pesquisa, a aplicação navega à página de busca de evento, e realiaza uma busca usando o texto inserido.
Note também que, ao realizar login, os \emph{links} para as páginas de login e cadastro são substitídos por elementos relacionados ao usuário, como um botão para realizar logout, um \emph{link} para a página de perfil, e um \emph{link} para a página de criar evento, no caso de um usuário Organizador.

% TEXTO HOMEPAGE

A página agenda é composta de dois modos diferentes: agenda diária e agenda semanal. 
A agenda diária é o modo inicial desta página, mas é possivel alterná-los por meio de botões.

No modo de agenda diária, é apresentado a data atual e uma grade, contendo todos os eventos que acontecem neste dia, organizados por sua hora de início. 
Nesta grade, são disposto ao longo de 24 colunas, correspondentes as 24 horas do dia.
Se alguma coluna contém eventos demais, é exibido um botão, que leva para a página de busca de eventos, onde é realizado uma busca que mostra todos os eventos realizados naquele horário.

Ao alternar para o modo de agenda semanal, é apresentado uma outra grade, onde são apresentados todos os eventos que ocorrem na semana da data atual, e também o domingo da próxima semana, a fim de exibir todo o final de semana.
Na grade da agenda semanal, cada coluna representa um dia da semana, e cada linha representa um horário.
Então, cada evento é exibido em uma célula correspondente a sua data e horário.
De maneira similar à agenda diária, uma célula que contém eventos demais exibe um botão, que leva para a página de busca de eventos, onde é realizado uma busca que mostra todos os eventos realizados naquela data e horário.
Adicionalmente, clicar na data de uma das colunas leva à agenda diária daquele dia. 

Clicar nos eventos apresentados nesta página dirige o usuário à página daquele evento. 
Há também duas caixas de seleção, onde é possível filtrar os eventos exibidos nesta página por suas categorias ou regiões. 
Este filtro tem efeito em ambos os modos.

A página de buscas permite buscar diretamente por eventos, passados e futuros, utilizando diversos filtros.
Além de navegar diretamente para esta página, é possível acessar ela por meio da barra de pesquisa, no cabecalho da aplicação, e também pela página de agenda.
Navegar para esta página desta maneira faz com que ela realize uma operação de busca com os parâmetros configurados pela página de onde se foi navegado.
Na coluna esquerda desta página, estão localizados diversos filtros: busca por texto (referente ao título e à descrição de um evento), por categoria, por região, por data, e por horário.

Vale notar que os filtros de horário se referem apenas ao horário do evento, e não a sua data. 
Da mesma forma, os filtros de data não levam em consideração o horário de um evento.
Este comportamento é deliberado, para possibilitar a filtragem de acordo com uma intervalo de horários e um intervalo de dias simultaneamente, ao invés de um grande intervalo de um momento até outro.

Realizar uma busca nesta página exibe uma lista de resultados, eventos cujas características se enquadram nos parâmetros informados pelo usuário.
Clicar em qualquer evento navega para a página de evento, onde são exibidas informações referentes a ele.

A página de evento, que é acessada por meio de divesas outras páginas da aplicação, exibe as informações de um determinado evento.
Além do nome do evento, são exibidos sua descrição, categoria, uma imagem ilustrativa, um \emph{link} para mais informações, a data e horário de início e de término do evento, sua região, e seu endereço.
Também são exibidas as atualizações deste evento, pequenos textos anexados ao evento, que descrevem eventuais atualizações sobre o evento (como, por exmeplo, um comunicado de troca de local).
É possível compartilhar o URL da página, direto do navegador, uma vez que a identificação do evento está presente nele. 

A última página acessível sem precisar realizar um login é justamente a página de login e cadastro.
Ambos os mecanismos de login e cadatro estão presentes nesta página, simultaneamente, a fim de evitar situações onde o usuário tenta acessar o cadastro e acaba acessando o login, ou vice-versa.
Ainda assim, a página destaca o componente de login ou cadastro, dependendo de qual \emph{link} foi clicado.

O componente de cadastro contém um formulário, com os campos nome, email, e senha, e enviá-lo efetua a criação de um novo usuário, do tipo Pessoa, no sistema. Também existe um botão que indica um modo de cadastro como Organizador. 
Clicá-lo faz com que enviar este formulário apresente um novo campo para um CPF ou CNPJ, e enviá-lo cria um novo usuário do tipo Organizador, ao invés de Pessoa.

O componente de login, presente na mesma página, permite realizar a operação de autenticação.
Enviar este formulário com um e-mail e senha válidos tem como resposta os dados do usuário referente, bem como um \emph{Token}, para que se autenticar ao acessar funcionalidades restritas.
Uma vez realizado o login, é possível acessar novas páginas, que requerem que o usuário esteja autenticado.
O conjunto de páginas disponibilizado para cada usuário varia de acordo com o seu tipo.

A página de perfil está disponível para todos os usuários autenticados, porém se comporta de maneira ligeiramente diferente, dependendo do tipo de usuário que a está acessando.
Todos os usuários podem acessar a página e conferir dados do seu perfil.

Acessar a página de perfil como um Organizador, além de exibir os dados do usuário, exibe também uma lista, de todos os eventos criados por este Organizador.
Para cada evento, são exibidas suas informações básicas, o seu status, e botões, que permitem editar e exclui-lo.
É possível ainda filtrar essa lista, pelo status dos eventos, para facilitar a visualização.

A página de edição de eventos, que permite editar ou atualizar um determinado evento, é acessível somente ao Organizador responsável pelo evento.
Esta página é dividida em dois componentes, um para adicionar atualizações ao evento, e outro para realizar uma edição no evento.
Eles estão agrupados nesta página porque ambos se referem a operações semelhantes: alterar o estado do evento.

O Componente de atualização de evento possui um formulário que, ao preenche-lo com um título e uma descrição, e envia-lo, adiciona uma atualização ao evento referente.
É possível ver esta atualização na página deste evento.

O Componente de edição, também presente nesta página, possui um formulário, com campos de descrição, link de mais informações, imagem, data e hora de início e fim, região, e endereço, todos referentes à atributos do evento. 
Durante o carregamento da página, estes campos são preenchidos com os dados originais deste evento.
Alterar algum campo deste formulário o marca com um tom de destaque, simbolizando que já nao reflete mais o valor original do evento.
O envio de dados deste formulário é constituido apenas pelos campos que foram alterados pelo usuário.
Vale notar que editar um evento o coloca de volta no estado de análise, e precisando que um Moderador o analise novamente e atualize seu status de acordo. 

Já a página de exclusão de eventos contém apenas alguns botões: voltar, editar e excluir, que realizam ações sob um determinado evento. 
Esta página também só é acessível ao Organizador deste evento.
O botão de voltar direciona o usuário de volta à página de perfil, e o botão de editar o direcona à página de edição deste mesmo evento.
O botão de excluir, por sua vez, elimina este evento da aplicação, tornando-o inacessível.

Por fim, a última nova págin a disponível a um Organizador é a de novo evento. 
Ela contém um formulário que coleta todas as informações necessárias para criar um novo evento, que são um nome, uma descrição, uma imagem ilustrativa, sua categoria, um link para mais informações, uma data e horário para início e término, o endereço e a região onde é realizado.
Ao preencher todos os dados de maneira válida e enviar o formulário, um novo evento é criado, mas não é imediatamente acessível: a princípio, possui um status de em análise, e um moderador deve analisar e aprová-lo para que possa aparecer na aplicação normalmente.
Todo evento, ao ser criado, tem a ele um Moderador designado automaticamente.

Ao fazer login como Moderador, é possível acessar a página de perfil e conferir que, além dos dados de usuário, é também exibida uma lista de eventos.
Os eventos desta lista são aqueles que tem este usuário como Moderador designado.
Nesta lista, além de informações básicas de cada evento, está seu status, e um botão, que permite realizar a análise do evento, caso o mesmo tenha o status em análise.
A lista está separada entre eventos já analisado e que devem ser analisa-dos. 

A página de análise de eventos é acessível ao Moderador designado de um determinado evento, por meio de seu perfil.
Nesta página, são exibidos todos os dados de um evento que está em análise, semelhante à página de eventos, de maneira que o Moderador possa o analisar.
Ao fim da página, há dois botões, que alteram o status deste evento para aprovado ou reprovado, além de um terceiro botão que leva de volta ao perfil.
Aprovar um evento significa que agora ele estará visível para todos os usuários da aplicação, enquanto reprovar um evento faz com que ele não fique disponível a outros usuários, exceto ao Organizador, que pode editá-lo para que o evento entre em análise novamente.
O julgamento do status de um evento, realizado pelo moderador, deve ser realizado de acordo com os termos de uso da plataforma.

Finalmente, existe uma página para qualquer URL não previsto pela aplicação.
Esta página também é exibida quando se tenta acessar um evento que não pode ser encontrado, ou quando se tenta acessar uma página restrita.

Falta as imagens de cada tela po eu vo por na proxima versao kkkkk
\chapter{Conclusão}
\label{ch.conclusao}

Este projeto mira em providenciar uma alternativa para consultar eventos e atividades culturais. 
Uma alternativa onde os usuários possam navegar e buscar por eventos culturais relevantes à eles, em uma única plataforma voltada para isso.
Uma alternativa one qualquer entidade organizadora de eventos possa divugar seus eventos, sem depender de mídias locais ou plataformas inadequadas.  

Por meio do trabalho desenvolvido, foi possível confirmar a factibilidade técnica desta plataforma, uma vez que foi implementado uma aplicação que cumpre os requisitos especifcados.
Lançar esta plataforma para uso geral, entretanto, é um desafio, que deve ser feito com o devido cuidado.

Tudo isso em prol de oferecer uma ferramenta que ajudasse as pessoas a encontrarem as atividades que tem interesse (ou até descobrir um novo!), e oferecer aos organizadores de eventos um público que aprecia os eventos que realizam. 
Desta forma, os organizadores de eventos teriam um aumento em seu público e em sua retenção, e as pessoas que frequentam esses eventos poderiam escolher o que mais lhe agrada.


% Este projeto mira ambiciosamente em se tornar a maneira mais usual de se consultar as atividades e eventos de uma determinada região, de maneira que os usuários utilizem-a como a primeira alternativa ao invés de navegar em páginas de redes sociais ou colunas de jornais locais. 
% A plataforma apresentaria dados de eventos de toda natureza, acompanhado de ferramentas para buscar, filtrar e organizá-los de maneira sucinta e simples. 
% Também agiria como uma plataforma para qualquer organização divulgar seu evento, não dependendo assim da redação dos jornais e mídias locais.

\section{Dificuldades encontradas}
\label{sec.dificuldades-encontradas}

Duran

\section{Trabalhos futuros}
\label{sec.trabalhos-futuros}

Embora o trabalho desenvolvido tenha atingido um estado satisfatório, ainda há espaço para desenvolver mais ideias, implementar melhorias e novas funcionalidades.

Nenhum sistema opera em um vácuo.
Isso significa que, para um eventual lançamento da plataforma desenvolvida, convém realizar uma análise de mercado profunda, para determinar a melhor estratégia de lançamento possível.
Isto inclui também uma campanha de marketing para a plataforma, para alcançar usuários e entidades organizadoras, bem como uma estratégia de monetização, levando em consideração os custos operacionais e maneiras de adquirir receita para sustentar sua operação e manutenção.

Outro componente importante para o funcionamento da plataforma são os termos de uso, documento que nortearia os Moderadores ao analisar os eventos da plataforma.
Ele deve ser elaborado com cuidado, a fim de garantir que o conteúdo da aplicação tenha qualidade, e que ela não seja pervertida por usuários maliciosos.

Conforme discutido neste trabalho, o desenvolvimento de uma versão que seja executável em aparelhos celulares e tablets tornaria a plataforma ainda mais acessível e prática.
Estar disponível neste tipo de aparelho tornaria a plataforma disponível a um público bem maior, e também possibilitaria acessá-la em momentos nos quais um computador com um e navegador e acesso à internet não estão facilmente acessíveis (como, por exemplo, enquanto se está na rua).

No que se refere à acessibilidade, a plataforma ainda não está de acordo com normas regulamentadoras sobre acessibilidade digital.
A fim de tornar a plataforma devidamente acessível à pessoas com deficiência, devem ser implementadas medidas que tornem o \emph{website} acessível. 
Estas medidas incluem, mas não estão limitadas a: um modo de alto contraste; suporte para libras; uso correto do texto alternativo; e legibilidade otimizada.

Visto que a aplicação é dividida entre cliente e servidor, seria necessário desenvolver apenas um novo componente cliente, que execute em dispositivos móveis, responsável por acessar os dados do servidor e apresentá-los ao usuário.
Adicionalmente, as tecnologias utilizadas no desenvolvimento do cliente para navegador web permitem que uma boa porção do código elaborado seja reaproveitado no desenvolvimento de um novo cliente, assim reduzindo o custo de desenvolvimento.

A principal melhoria ao sistema que poderia ser desenvolvida seria o emprego de uma integração com o Google Maps.
O serviço de mapas oferecido pela Google permite a interação com um mapa virtual, onde é possível navegar, escolher endereços, marcar e exibir lugares específicos.
A implementação desta integração consistiria em recolher um lugar no mapa ao cadastrar um novo evento, correspondente ao endereço do mesmo, e também na exibição deste lugar na página de eventos.
Desta maneira, os usuários podem conferir o endereço do evento em um mapa, definirem uma rota de sua localização ao local do evento, e todas as outras funcionalidades que o Google Maps já oferece.

Outras funcionalidades envolvendo o serviço de mapas inclúem a filtragem de eventos por geolocalização na página de busca, e também uma nova página, contendo um mapa onde se exibem eventos que acontecem numa determinada região.

% AFERIR HOMEPAGE
Embora a página de agenda cumpra este papel, o desenvolvimento de uma \emph{homepage}, com o intuito de ser a página principal da aplicação, pode torná-la mais agradavel de se utilizar. 
A mesma poderia exibir eventos em destaque, e eventos relevantes ao usuário, levando em conta a data, hora e região de acesso. 
Ter uma página assim como a primeira que um novo usuário encontraria deve transmitir a ideia de como a aplicação e seus eventos funcionam, tornando o \emph{onboarding} mais suave e aumentando as chances do usuário voltar a acessar a aplicação.

Ao realizar o cadastro, a conta do usuário é automaticamente ativada. 
Idealmente, seria implementado um sistema de verificação de email para realizar o login, bem como um sistema de recuperação de senha, caso o usuário a esqueça.

Ao cadastrar um evento, seria ideal recolher a classificação indicativa do mesmo, e utilizar este parâmetro como filtro nas páginas de busca de eventos e agenda de eventos. 
Esta é uma informação importante especialmente para pais e responsáveis, e ter mais este atributo relacionado aos eventos tornaria a experiência deste grupo ainda melhor.

Em relação aos atributos de um evento, permitir o upload de mais imagens, imagens maiores, e títulos e descrições mais longas podem tornar o uso das páginas de criação e edição de eventos mais flexível e agradável aos organizadores, tornando os eventos mais expressivos e detalhados.

Tanto a solução de armazenamento de imagens, quanto o mecanismo de designação de moderadores a eventos foram desenvolvidas com a intenção de serem melhoradas conforme a aplicação crescer e escalar.
Mais especificamente, o armazendamento de imagens é feito no próprio servidor, mas o serviço que gerencia o armazenamento pode ser substituído por um novo serviço, que utilize de armazenamento em nuvem, por exemplo.
De maneira similar, a função que atribui um moderador à um evento o faz de maneira aleatória, mas pode ser substituída para realizar uma escolha mais informada.
Estas funcionalidades foram desenvolvidas tendo em mente o princípio do baixo acoplamento, o que significa que podem ser alteradas sem comprometer o funcionamento de outras partes da aplicação.
Isso não significa que estas duas funcionalidades foram implementadas de maneira inadequada, mas sim que elas funcionam para uma escala específica da aplicação.

Ao longo da aplicação, cabem diversas pequenas otimizações, que tornariam a experiência de utilização mais agradável.
Estas inclúem: uma maneira de pré-visualizar a página de evento, a partir das páginas de novo evento e editar evento; deixar registrado preferências de usuários com filtros na página de agenda de eventos, a fim de mostrar eventos mais relevantes; e oferecer uma opção de ignorar eventos já passados na página de busca de eventos.

% \chapter{Conclusão}
\label{ch.conclusao}



Lorem ipsum dolor sit amet, consectetur adipiscing elit. Nunc convallis odio nec dapibus elementum. Sed lectus mauris, imperdiet consectetur volutpat nec, molestie quis dui. Etiam est magna, porta eu convallis vitae, ultrices in tortor. Mauris eu justo felis. Mauris cursus sed mauris id semper. Nulla facilisi. Proin sollicitudin fermentum orci. Ut bibendum faucibus libero ac mollis. Curabitur eget dignissim lectus.

Phasellus vitae ante vitae mi aliquet rhoncus. Cras luctus, elit quis laoreet ultrices, arcu purus rhoncus massa, ut venenatis lectus mauris id nunc. Vivamus ut ex nec velit aliquet faucibus. Aenean ligula purus, dignissim in volutpat a, egestas vel elit. Nunc vel quam nec libero volutpat accumsan. Maecenas eget velit convallis, pulvinar quam lobortis, mattis metus. Mauris feugiat tellus id arcu convallis, nec consequat lorem aliquam. Pellentesque vestibulum neque nec rhoncus gravida. Etiam at congue libero. Etiam sagittis malesuada accumsan. Aliquam imperdiet enim et scelerisque semper. Duis vitae lacus iaculis, consequat nisi at, dapibus magna. Donec urna tortor, ultricies at nibh vel, dictum gravida velit. Donec magna risus, vestibulum eu eleifend sit amet, varius vestibulum quam. Mauris euismod at ligula id imperdiet. Sed at velit sed felis pharetra imperdiet at et nulla.

Phasellus sodales at ex vitae eleifend. Praesent purus purus, vestibulum vel sapien ut, malesuada mattis nulla. Morbi tellus nibh, aliquam at risus eu, aliquam aliquet lorem. In hac habitasse platea dictumst. Nullam ac fermentum tellus, sit amet semper metus. Donec eu orci libero. Maecenas metus orci, hendrerit quis aliquet non, laoreet ac massa. Praesent ullamcorper egestas erat et porttitor. Nunc lectus ante, dignissim sit amet mi eget, ultricies vehicula ligula. Aliquam semper libero vitae cursus malesuada. In fringilla dictum purus eget feugiat. Nullam tristique tristique ipsum, a fermentum nulla tempus eget. Proin ultrices metus vel massa luctus, aliquam tincidunt ante sollicitudin. Phasellus turpis orci, pellentesque ut mi et, lacinia congue turpis. Curabitur semper pellentesque nisi, in elementum metus dignissim sit amet.

Curabitur cursus tellus nec risus imperdiet, in fermentum elit aliquet. Sed velit ex, finibus vitae pretium lobortis, viverra et justo. Integer sed imperdiet turpis. Nam efficitur aliquam tincidunt. Vestibulum tortor metus, interdum in pretium sit amet, ultricies in odio. Duis aliquet mattis diam tincidunt sollicitudin. Curabitur ultricies, risus eget imperdiet congue, nunc lectus lobortis lorem, et congue quam est ut ipsum. Sed ullamcorper tellus vel interdum gravida. Sed dapibus erat nunc, sit amet placerat dolor tempus quis. Duis pulvinar, ante sed tempor aliquet, orci lacus tempor libero, mollis pellentesque tortor nunc tempus nulla. Duis finibus condimentum metus ac cursus.

Pellentesque risus arcu, tincidunt id metus vel, congue semper urna. Aenean et ligula lacinia magna eleifend iaculis. Donec ante metus, hendrerit eget neque at, consectetur auctor sem. Vivamus leo velit, sollicitudin a tincidunt vitae, feugiat vel elit. Integer eleifend lacus nisi, a imperdiet lorem bibendum at. In ut fringilla lectus. Aliquam consequat enim magna, id consectetur dui tincidunt ullamcorper. Duis tincidunt et mauris sit amet lobortis. Aliquam augue ante, aliquam quis suscipit vel, consectetur a leo. In ex lorem, lobortis sit amet tristique eget, gravida ut justo. Nulla tempor tellus nisi, quis dapibus sem blandit quis. Nullam non nulla vulputate, efficitur eros sit amet, auctor urna. Nunc vestibulum sem et orci luctus, pharetra porta quam aliquam. Nunc et faucibus leo, vitae luctus est. 


% --------------------------------------------------------
% ELEMENTOS PÓS-TEXTUAIS
% --------------------------------------------------------

\postextual


% --------------------------------------------------------
% REFERÊNCIAS BIBLIOGRÁFICAS
% --------------------------------------------------------


\nocite{postgresql}
\nocite{java}
\nocite{springboot}
\nocite{angular}
\nocite{typescript}
\nocite{jwt}
\nocite{cliente-servidor}
\nocite{http}
\nocite{rest}
\nocite{pwa}
\nocite{ionic}

\bibliography{referencias}


% --------------------------------------------------------
% GLOSSÁRIO
% --------------------------------------------------------

% Consulte o manual da classe abntex2 para orientações sobre o glossário.
%\glossary


% --------------------------------------------------------
% APÊNDICES
% --------------------------------------------------------

% TODO: verificar apendices

% % Inicia os apêndices
% \begin{apendicesenv}
% %Imprime uma página indicando o início dos apêndices
% \partapendices
% \label{apendices}

% \chapter{Código de processamento do MediaPipe e geração dos arquivos JSON}
% \label{apend.codigo-python}

% \lstinputlisting[label={cod.processa-mediapipe}, language=Python]{code/teste_saida_mp.py}

% \chapter{Código de animação do esqueleto no Unity3D}
% \label{apend.codigo-csharp}

% \lstinputlisting[label={cod.anima-esqueleto}, language=C]{code/MediaPipeJSONParser.cs}

% \end{apendicesenv}


% --------------------------------------------------------
% ÍNDICE REMISSIVO
% --------------------------------------------------------

%\printindex


% --------------------------------------------------------
% FINAL DO DOCUMENTO
% --------------------------------------------------------

\end{document}